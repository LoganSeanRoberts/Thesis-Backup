\chapter{Fitting Correlation Functions}\label{cha:fitting_correlation_functions}
This chapter will likely be one of the more substantial sections in my thesis.  In chapters \ref{DstopiK} and \ref{cha:BstopiK} I will discuss my findings for $B_{(s)}\rightarrow \pi(K)$ and then $D_{(s)}\rightarrow \pi(K)$.  These four different meson decays are all built on the same lattice data sets to which I have access.  As a result, the same methodology in fitting various correlation functions is largely shared among these four meson decays.  The section outline for this chapter will likely go as follows:
\begin{itemize}
    \item \textbf{Lattice units}.  In much of the subsequent math to follow in this chapter, we work in lattice units for masses and energies.  Therefore masses $M$ and energies $E$ in equations are actually $aM$ and $aE$.  The instances in which this will not be true are where we talk explicitly in physical units of GeV.  
    \item \textbf{Correlation Function Calculation}. Here is where I intend to establish from first principles the characteristics of two and three-point correlation functions.  It will be necessary to discuss phases in our staggered quarks, spin-taste structure, creation and annihilation operators, and current insertions.  This will be followed by discussion of sources and sinks on the lattice, random domain walls, and twisted boundary conditions.
    \item \textbf{Correlator Fit functions}.  Here we get into the realm of Lattice QCD which which I am most familiar as of writing.  I can introduce the classic two and three-point correlator fit functions, equations \ref{eq:2ptcorrfit} and \ref{eq:3ptcorrfit}.  It would also be conducive to include some sample log(correlator) plots.
\end{itemize}
\begin{equation}\label{eq:2ptcorrfit}
    C_2^\mathcal{M}(t)=\sum^{N_\mathrm{exp}}_{i=0}\big(|A_i^{\mathcal{M},\mathrm{n}}|^2(e^{-E_i^{\mathcal{M},\mathrm{n}}t}+e^{-E_i^{\mathcal{M},\mathrm{n}}(N_t-t)}) -(-1)^{t}|A_i^{\mathcal{M},\mathrm{o}}|^2(e^{-E_i^{\mathcal{M},\mathrm{o}}t}+e^{-E_i^{\mathcal{M},\mathrm{o}}(N_t-t)})\big).
\end{equation}
\begin{equation}\label{eq:3ptcorrfit}
  \begin{split}
    &C^{\mathcal{M}_1,\mathcal{M}_2}_3(t,T)=\sum^{N_\mathrm{exp}}_{i,j=0}\big(A_i^{\mathcal{M}_1,\mathrm{n}}J_{ij}^{\mathrm{nn}}A_j^{\mathcal{M}_2,\mathrm{n}}e^{-E_i^{\mathcal{M}_1,\mathrm{n}}t}e^{-E_j^{\mathcal{M}_2,\mathrm{n}}(T-t)}\\
    &-(-1)^{(T-t)}A_i^{\mathcal{M}_1,\mathrm{n}}J_{ij}^{\mathrm{no}}A_j^{\mathcal{M}_2,\mathrm{o}}e^{-E_i^{\mathcal{M}_1,\mathrm{n}}t}e^{-E_j^{\mathcal{M}_2,\mathrm{o}}(T-t)}\\
    &-(-1)^{t}A_i^{\mathcal{M}_1,\mathrm{o}}J_{ij}^{\mathrm{on}}A_j^{\mathcal{M}_2,\mathrm{n}}e^{-E_i^{\mathcal{M}_1,\mathrm{o}}t}e^{-E_j^{\mathcal{M}_2,\mathrm{n}}(T-t)}\\
    &+(-1)^{T}A_i^{\mathcal{M}_1,\mathrm{o}}J_{ij}^{\mathrm{oo}}A_j^{\mathcal{M}_2,\mathrm{o}}e^{-E_i^{\mathcal{M}_1,\mathrm{o}}t}e^{-E_j^{\mathcal{M}_2,\mathrm{o}}(T-t)}\big).
  \end{split}
\end{equation}
\begin{itemize}
    \item \textbf{Bayesian Fitting} This might be a shorter section, but detailing our Bayesian fitting approach likely warrants its own section.  This section will briefly summarize why a correlator fitting methodology which involves Bayesian fitting, as outlined in \textit{constrained curve fitting} does not compromise the statistics of our work.  Naturally this discussion introduces the idea of gaussian priors, which neatly leads us to the next section.
    \item \textbf{Prior determination methodology}. I have done quite a lot of work on this subject already.  This section will use equations \ref{eq:chi2_prior_expression}-\ref{eq:3ptVeff} in a discussion of effective mass and amplitude plots, and log(GBF) optimization.  
\end{itemize}
\begin{equation}\label{eq:chi2_prior_expression}
    \chi_{\text{prior}}^2 = \sum_i  \frac{(E_i - \Tilde{E}_i)^2}{(\Tilde{\sigma}_{E_i})^2} + \sum_i \frac{(A_i - \Tilde{A}_i)^2}{(\Tilde{\sigma}_{A_i})^2} + \sum_{i}\frac{(J_{i} - \Tilde{J}_{i})^2}{(\Tilde{\sigma}_{J_{i}})^2}.
\end{equation}
\begin{align} \label{eq:2ptMeff}
        M_{\text{eff}}(t) &= \frac{1}{2}\cosh^{-1}\left(\frac{C_2(t-2)+C_2(t+2)}{2C_2(t)}\right) \\
        \label{eq:2ptAeff}
        A_{\text{eff}}(t) &= \sqrt{\frac{C_2(t)}{e^{-M_{\text{eff}}t}+e^{-M_{\text{eff}}(N_t-t)}}} \\
        \label{eq:3ptVeff}
        J_{\text{eff}}(t,T) &= C_3(t,T)\frac{\exp[{M_{\text{eff}}^{\pi,K}t +M_{\text{eff}}^{H_{(s)}}(T-t)}]}{A_{\text{eff}}^{\pi,K} \times A_{\text{eff}}^{H_{(s)}}}
        \end{align}
\begin{itemize}
     \item \textbf{SVD Cuts}.  This might be a relatively smaller section, but it will be important to discuss the reality of having more data than samples, the covariance matrix and eigenvalues.
    \item \textbf{Determining Goodness of Fit}.  Here I will describe tools we use to determine the goodness of any particular fit.  These include evaluating $\chi^2/\text{d.o.f.}$, and log(GBF).  I would likely include a section on SVD and prior noise as well, with reference to the artificial lowering of $\chi^2/\text{d.o.f.}$ that comes from wide priors and SVD cuts.
\end{itemize}

\section{Introduction to priors for correlator fitting}
\subsection{The case for using priors}
For this series of correlator fits, I provide \textit{a priori}, or prior estimates for posterior fit outputs.  These priors augment the standard $\chi^2$ fitting procedure (equations \ref{eq:augmented_chi2} and \ref{eq:chi2_prior_expression})  The inclusion of priors in my fitting methods follows the methodology first outlined in \cite{bayesian}, and I provide an effective summation and justification here.  In the general fitting process of a meson two-point correlator, we attempt to extract a set of energies $E_n$ and amplitudes $A_n$ from the correlator (or set of correlators)  $C_2(t)$, which we can write generally as
\begin{equation}\label{eq:meson_correlator}
    C_2(t) = \sum_{n=0}^\infty A_n e^{-E_nt}.
\end{equation}
In this common scenario, we are troubled with managing an infinite stacked tower of $n^{th}$ order energies and amplitudes with correlators of finite temporal length (which for this project range between $t_{\text{max}/a} = 96$ to $t_{\text{max}/a} = 192$).  

In \cite{bayesian} the host of problems that result from this reality are mitigated by the introduction of \textit{constrained curve fitting}.  In this method, we introduce an \textit{a priori} estimate for the mean and standard deviation\footnote{The default prior distribution choice is a Gaussian. In section 3.1 of \cite{bayesian} the arbitrary nature of this distribution choice is detailed more fully.} of a given $E_n$ or $A_n$ in the fitting procedure, a set of which we can collectively call \textit{the priors} of a given fit.  This method, among other things, enforces realistic physical behavior of higher order energies and amplitudes.  Following from equations 5 and 6 in \cite{bayesian}, we augment the normal $\chi^2$ fitting procedure:
\begin{equation}\label{eq:augmented_chi2}
    \chi^2 \rightarrow \chi^2_\text{aug} = \chi^2 + \chi_{\text{prior}}^2.
\end{equation}
We can denote the prior on a given $E_n$ or $A_n$ as $\Tilde{E_n} \pm \Tilde{\sigma}_{E_n}$ or $\Tilde{A}_n \pm \Tilde{\sigma}_{A_n}$ respectively.  This defines $\chi_{\text{prior}}^2$ as
\begin{equation}\label{eq:chi2_prior_expression}
    \chi_{\text{prior}}^2 = \sum_n  \frac{(E_n - \Tilde{E}_n)^2}{(\Tilde{\sigma}_{E_n})^2} + \sum_n \frac{(A_n - \Tilde{A}_n)^2}{(\Tilde{\sigma}_{A_n})^2}.
\end{equation}

In the generation of correlators (from $N_f = 2 + 1 + 1$ MILC-HISQ gluon fields \cite{MILC_2010, MILC_2012}), we impose that the amplitudes $A_n$ are "well behaved" in that they have non-infinite values.  This imposes a behavior of rapid decay among higher order states along the temporal lattice length.  This allows us to begin fitting at some $t=t_\text{min}$, where the contributions to the correlator data at $t \geq t_\text{min}$ are dominated by the first few and lowest ordered states.  Adjusting $t_\text{min}$ comes with the trade-off of different systematic errors, but most importantly, it reduces the effective number of priors for each state (a value of $n_\text{max} = 4$ has been common throughout this project).

In tables \ref{tab:2pt_priors} and \ref{tab:3pt_priors}, I give a (working and mostly comprehensive) list of priors that go into my fitting script.  In the following sections I go into greater detail explaining what these priors are, how they are decided upon, and which priors can be determined with computational methods. 

\subsection{Prior determination methodology}\label{sec:prior_determination_methodology}
The methods I use to determine the priors for these correlator fits fall into two general categories: manual and automated.  Effective amplitude plots (such as figure \ref{fig:3pt_amps_fine}) can be constructed via equations \ref{eq:2ptaeff} and \ref{eq:3ptVeff} from two-point and three-point correlators respectively.  So long as the correlator data from which these effective amplitude plots are constructed are not too noisy, ground state behavior can be visually (which is to say, manually) differentiated from higher order and oscillating states. 

Some effective amplitude plots (and effective mass plots in the case of two point correlators \ref{fig:aMeff_sample}) are best determined algorithmically.  In such cases where a ground state plateau is distinct enough from non-ground state signal, a function can take the rolling average of adjacent and consecutive $A_{\text{eff}}(t)$ and $J_{\text{eff}}(t,T)$ values.  The central value of the ground state amplitude (or mass) can be determined from where the change in this rolling average is minimized.

Irrespective of methodology, priors are normally set conservatively.  In other words, a prior's uncertainty (standard deviation) is tuned such that its magnitude is no less than ten times its corresponding posterior's uncertainty.  As a general guiding principle, we aim to avoid using priors as strict constraints on posteriors.  We take exception to this principle when we have additional physics information to which we expect the standard model-derived correlator data to adhere.  Sections \ref{sec:dispertion_relation} and \ref{sec:excited_state_energies} details examples of this in relation to the dispersion relation and excited state meson energies respectively.  In such cases, physics based algorithms are used to automatically assign values to these priors.

Finally, in the cases where we cannot use effective amplitude plots, nor physics based preconceptions to determine priors, we utilize GBF optimization. Much more information on the Bayes factor can be found in \cite{sivia2006data, carlin2010bayes}, but to effectively summarize its use to this fitting procedure: prior tuning that maximizes a fit's GBF indicates that the prior neither over-constrains nor under-constrains the fit \cite{bayes_Kass_Rafferty}.  For priors indeterminable by the previously noted methods, such as oscillating excited three point amplitudes, we opt to use this method.  A prior is GBF optimized by performing a test fit, and recording the fit's GBF (or the log of its GBF).  Then, the prior's mean or standard deviation is slightly adjusted.  The fit is performed once more and the subsequent change (if any) of GBF is recorded.  This process is repeated until a clear maximum is resolvable. The GBF optimization of a prior's uncertainty is aborted however if maximizing a fits GBF violates the 10:1 prior to posterior magnitude ratio principle.  In such cases, more conservative (larger standard deviation) priors are chosen over more precise, GBF-preferred priors.

\section{Two-point correlation function priors}
In the following subsections, I discuss the priors on meson amplitudes and energies.  These priors are used in equation \ref{eq:chi2_prior_expression} to fit two point correlator functions (equation \ref{eq:meson_correlator}).  These priors cover the whole range of two-point correlator related fit parameters. This includes the ground state and higher order exponential state energies and amplitudes of: mother mesons with varying masses and spin-taste structures, daughter mesons of varying momentum, and the daughter meson oscillating state counterparts.

%For each set two-point correlators that correspond to a $B_{(s)}$ of mass $am_b$, there are four correlator subsets for the four spin-taste copies 

\subsection{Meson rest masses}\label{sec:meson_rest_masses}
The first set of priors we can discuss are the rest masses of the mother and daughter mesons.  By construction, the heavy mother mesons in this work are in the rest frame of the lattice, and so their energies are equivalent to their masses.  The lighter daughter mesons however are simulated at varying momenta following the twisted boundary methodology from \cite{Bedaque_2004, Guadagnoli_2006}.  This methodology relates some twist angle $\theta$ to a meson's three-momentum (in lattice units) $|a\overrightarrow{p}|$:
\begin{equation}
    \theta = |a\overrightarrow{p}| \times \frac{N_x}{\sqrt{3}\pi},
\end{equation}
where $N_x$ is the lattice unit length in the 

In this formulation, a daughter meson of \textit{zero-twist} corresponds to a $q^2_{\text{max}}$ current insertion; the daughter meson remains at rest in the lattice frame and its energy equals its rest mass.

For a two-point correlator $C_2(t)$, we can plot its effective mass $aM_{\text{eff}}(t)$ via equation \ref{eq:2ptam}.  For a well behaved\footnote{Which is to say, the higher order exponential states rapidly decay.} two-point correlator, it is possible to extract an estimate for its rest mass from its effective mass plot.  
\begin{equation} \label{eq:2ptam}
        aM_{\text{eff}}(t) = \frac{1}{2}\cosh^{-1}\left(\frac{C_2(t-2)+C_2(t+2)}{2C_2(t)}\right). 
\end{equation}
Figure \ref{fig:aMeff_sample} shows sample effective mass plots for $H\rightarrow \pi$.  From this figure we can see that the effective mass function follow a trend where, after $t_{\text{min}}$, the higher order states decay quickly.  Then, the data stabilizes with relatively small uncertainty and minimal central value fluctuations.  This is the signal for the ground state rest mass.  Then, the uncertainty and the size of central value fluctuations grows towards the middle of the lattice length.



This consistent behavior (a steady region of relatively noiseless ground state energy/rest mass signal) in the effective mass plot permits the use of an automated prior determination method.  WE utilize a simple \verb|Python| function which takes a rolling average over sets of 4 adjacent effective mass values (that is to say: at 4 consecutive lattice time slices $t/a$) and finds where this rolling average changes the least.  The central value of the rest mass's prior is then assigned to be equal to change-minimized, rolling averaged $aM_{\text{eff}}(t)$.  Figure \ref{fig:effective_mass_plots}

\begin{figure}
    \centering
    \caption{Sample pion effective mass plots on the f5 and uf5 ensembles.  The effective mass $aM\mathrm{eff}(t)$, calculated from equation \ref{eq:2ptMeff} and its respective two point correlation function $C^\pi_2$ is shown in blue. The rolling average of the effective mass R.Avg$(t)$ is shown in dashed green. Where the change in this rolling average is minimized $\Delta_\mathrm{min}$ of R.Avg$(t)$ is shown in dotted red. Consecutive red bands show regions of $\pm5\%$, out to $\pm30\%$, of $\Delta_\mathrm{min}$ of R.Avg$(t)$.}
    \includegraphics[width=0.75\linewidth]{chapter4/pdfs/Effective_Mass_Plots.pdf}
    \label{fig:effective_mass_plots}
\end{figure}

We then assign it a conservative uncertainty (which has practically been in the range of $5\%-10\%$\footnote{In many cases, we observe that a rest mass's prior uncertainty of $5\%$ is several orders of magnitude larger than its corresponding posterior uncertainty in test fits.}, such that in test fits, the prior uncertainty 10:1 ratio rule of thumb discussed in section \ref{sec:prior_determination_methodology}.  In practice, this method is incredibly effective at producing priors that accommodate the corresponding fit posteriors.  [UNFINISHED SECTION - MASS SCALING TERM?]
 
%Table \ref{tab:aMeff_values} shows the prior and posterior estimates for the same masses displayed in figure \ref{fig:aMeff_sample}.  From this table, we can see that the fit posteriors for these masses (zero-twist energies) have significantly smaller percent errors than the prior percent errors.  For this reason, I consistently use this automated method to determine priors for meson masses.

%The above method works for both the mother and daughter mesons in this study.  

From these ground state, zero-twist, non-oscillating meson energies, we can determine priors for their ground state, zero-twist, \textit{oscillating} counterparts.  The ground state daughter mesons have angular momentum and parity quantum numbers of $J^P = 0^-$.  The lowest lying, non-ground state is the same as the ground state but with opposite parity: $J^P = 0^+$.  With the HISQ action \cite{HISQ:2006rc}, this opposite parity manifests as oscillating propagators on the lattice, and therefore it is referred to as the "lowest lying oscillating" state.  Unfortunately, these cannot be calculated with the same method as before.  Instead, we rely on the known mass splitting between the non-oscillating state and lowest oscillating state given by the Particle Data Group in \cite{Workman:2022ynf}. For Example, the ground state pion has a rest mass of about 140 GeV, while the opposite parity state has a rest has a mass of roughly 500 GeV.  So, we set an estimate for the lowest energy oscillating mass by adding some fixed mass value to the ground state mass.  We then follow the same procedure for the kaon.  Appropriately, we assign a broader prior width to this rougher estimate ($20\%$ for example).  In multiple test fits, the posterior errors on these oscillating energies do roughly appear to be about one order of magnitude larger than those of the ground state energies (rest masses).  


\subsection{Non-zero-twist meson energies: the dispersion relation}\label{sec:dispertion_relation}
From a set of pion and kaon zero twist energy (rest mass) priors, we can use the dispersion relation to determine the priors for non-zero twist pion and kaon energies.  On the lattice, we modify the traditional, continuum limit dispersion relation with a term that accounts for discretization effects.  With the HISQ action \cite{HISQ:2006rc}, the lowest order discretization effects are of the order $(a\overrightarrow{p})^2$, where $a$ is the lattice constant and $\overrightarrow{p}$ is the meson's 3-momentum.  Following from \cite{Will_technical} and \cite{Chakraborty_2021}, we can then write the ground state ($n=0$), non-zero twist energy priors $P[aE^{\pi,K}_{0,\overrightarrow{p}}]$  as
\begin{equation}\label{eq:dispersion_energy}
   P[aE^{\pi,K}_{0,\overrightarrow{p}}] = \sqrt{P[aE^{\pi,K}_{0,\overrightarrow{0}}]^2 + (a\overrightarrow{p})^2}
   \times \left[ 1 + P[\epsilon_{\pi,K}] \left( \frac{a\overrightarrow{p}}{\pi} \right)^2     \right],
\end{equation}

Here, the parameters $\epsilon_{\pi,K}$ scale the relative size of discretization effects on the dispersion relation.  Different discretization parameters are introduced for either \BtoPi\space or \BstoK\space to account for any daughter meson mass dependency.  These discretization parameters are also given priors in the fitting procedure.  Following the findings from comparable HISQ-action fits \cite{Bouchard_2013, Will_towardsB, Chakraborty_2021, Will_technical}, the priors for $\epsilon_{\pi,K}$ are set as $P[\epsilon_{\pi,K}] = 0\pm1$.  In various test fits for this work, the posteriors for $\epsilon_{\pi,K}$ fall within the bounds of these priors.

We can use this same dispersion relation method to set priors for non-zero twist energies of the lowest lying oscillating (order $n=0$) daughter meson states.  In these cases we use a a separate prior for the parameter $\epsilon_\text{osc}$ whose use and prior value is identical to $\epsilon_{\pi,K}$ in equation \ref{eq:dispersion_energy}: $P[\epsilon_\text{osc}] = P[\epsilon_{\pi,K}] = 0 \pm 1$. The discretization parameter $\epsilon_\text{osc}$ is only used for the generation of non-zero twist, oscillating \textit{kaon} energy priors via the dispersion relation.  This is because there is no oscillating pion state with zero twist off of which to base the dispersion relation [CITATION NEEDED].  For this reason, non-zero twist oscillating pion energies are given priors constructed as: [WRONG, CHECK FUNCTIONS SCRIPT WHEN AVAILABLE]
\begin{equation}\label{eq:pion_prior_oscillating_energy}
P[aE^{\pi_{\text{osc}}}_{0,\overrightarrow{p}}] = \sqrt{\text{osc pion mass}^2 + (a\overrightarrow{p})^2}
\end{equation}

It is important to note that $\epsilon$'s use in this fitting procedure is limited only to the construction on non-zero twist meson energy priors.  In other words, $\epsilon$ does not \textit{directly} constrain a fit's posterior value for $aE^{\pi,K}_{0,\overrightarrow{p}}$.  In the fitting procedure, various $aE^{\pi,K}_{0,\overrightarrow{p}}$ are fit no differently than any other fit parameter (equation \ref{eq:chi2_prior_expression}). 

INSERT STUDY OF LHS VS RHS DISPERSION TESTING


Following this equation is the chosen automated method for determining these specific priors.


\subsection{Excited state meson energies}\label{sec:excited_state_energies}
In the previous subsections, we discuss prior determination of all $n=0$ meson (two-point) energy states.  The this section we address the collection of $n \neq 0$ states.  Referring back to equation \ref{eq:meson_correlator}, there are in theory an infinite number of higher energy states.  Fortunately, these higher order states are constructed such that they are "well behaved" and they decay much more quickly along the temporal length of the lattice.  The effective mass plots in figure \ref{fig:aMeff_sample} shows this behavior: the higher order states decay and the data is dominated by the ground state at lattice time slices $10<t/a<20$.

With an appropriate value for $t_\text{min}$, we can estimate that the influence of all higher order states of $n > n_\text{max}$ can be either ignored, or be thought of as "folded into" the signal of the $n = n_\text{max}$ state.  In other words, we can truncate the infinite equation (equation \ref{eq:meson_correlator}) to some small number of exponentials $n_\text{max}$.

In various test fits, we find that a number of exponentials $n_\text{max} = 4$ maximizes \verb|log(GBF)| and minimizes $\chi^2/\text{d.o.f}$\footnote{Choosing $n_\text{max}$ warrants investigation beyond GBF and $\chi^2/\text{d.o.f}$ optimization; a separate journal entry in the future will likely cover it}.

Thankfully, the structure of lattice QCD provides us a simple technique in determining the priors on these higher order energies.  The natural energy scale of QCD we call $\Lambda_\text{QCD}$, and for this research $\Lambda_\text{QCD} \approx 500$ GeV.  The quantized nature of these different energy levels allows us to estimate that for any $E_n$, $E_{n+1} = E_n + \Lambda_\text{QCD}$.  This simple relation allows us to easily set priors for exited state energies, and therefor is the method we use.  This is done practically by defining a general energy difference prior $\Delta E$ which is equal to $\Lambda_\text{QCD}$, and assigning it an uncertainty of $50\%$, so as to not over-constrain the fitting procedure  In fact, an assigned uncertainty of $50\%$ covers the entire theoretical energy range despite the expectation of energy level quantization.

%$\Lambda_\text{QCD}$ as defined for use in prior calculation is a \verb|gvar| object with a broad width equal to half its central value. 

\subsection{Ground state meson amplitudes}\label{sec:2pt_amps_ground_state}
%As alluded to in section \ref{sec:dispertion_relation}, rather than relying on automated methods like the dispersion relation, I instead manually evaluate effective amplitude plots to determine two-point amplitude priors.  
Like the effective mass plots discussed previously, we can construct effective amplitude plots for a two point correlator $C_2(t)$ via the following equation:
 \begin{equation}\label{eq:2ptaeff}
    A_{\text{eff}}(t) = \sqrt{\frac{C_2(t)}{e^{-M_{\text{eff}}t}+e^{-M_{\text{eff}}(N_t-t)}}},
\end{equation}
where $N_t$ is the temporal length of the lattice in lattice units, and $M_\text{eff}$ is the effective mass at lattice time slice $t$.  Figure \ref{fig:2pt_amps_fine} gives an example of overlaid mother meson effective amplitudes of various mass and twist combinations.

%These two-point amplitudes also have oscillating and higher order counterparts, however the determination of these is not easily done simply by evaluating effective amplitude plots such as figure \ref{fig:2pt_amps_fine}.  In this figure, the effective amplitudes of both the mother and daughter mesons, for all mass and twist combinations, are presented for a single ensemble.  From plots such as this, the only prior we can determine from "eye-balling it" is a non-oscillating ground state amplitude for both mesons.  %Additionally, its evident that the \BtoPi\space and \BstoK\space effective amplitude plots on the same ensemble demonstrate similar behaviors: a effective amplitude that stabilizes just before $t/a = 10$ at about $A_\text{eff} = 0.250$.  


Using this effective amplitude function, we can use a similar algorithm to the one discussed in section \ref{sec:meson_rest_masses} to find where the rolling average of consecutive values of $A_{\text{eff}}(t)$ changes the least.  
NEED TO DOUBLE CHECK HOW THIS ACTUALLY WORKS

In testing, we found that in the minority of cases, some algorithmically determined amplitude priors are given uncertainties considerably larger than their central values.  In other words, this automated method may assign an amplitude prior of $0.2 \pm 0.4$, wherein the uncertainty is twice the size of the magnitude of the central value.  Our two-point correlators are generated such that their ground state amplitudes are always positive [CITATION NEEDED], and so we want priors that discourage negative central values for ground state amplitudes.  In such cases, we override the algorithmically derived prior with a more conservative, negative central value discouraging prior that is manually determined from effective amplitude plots.  This manual override prior is broad enough to encompass the reasonable\footnote{Reasonable here denotes the plateau range in effective amplitude plots that occurs after the excited state signal has decayed, and before the signal noise becomes to great nearer the middle of the lattice length.} effective amplitude ranges of all mass-twist combinations for all mother and daughter mesons on a single ensemble. 


\subsection{Exited and non-zero twist meson amplitudes}

Like the non-zero twist energies of daughter mesons, we can use a similar dispersion relation as a physics-informed prior determination method for the amplitudes of non-zero twist daughter mesons. In equation \ref{eq:dispersion_amplitude}, we construct the prior for a non-zero twist, $n=0$ state daughter meson amplitude $P[A^{\pi,K}_{0,\overrightarrow{p}}]$ from the energy and amplitude of its zero twist counterpart.

\begin{equation}\label{eq:dispersion_amplitude}
P[A^{\pi,K}_{0,\overrightarrow{p}}] = \frac{P[A^{\pi,K}_{0,\overrightarrow{0}}]}{[1+(a\overrightarrow{p}/P[aE^{\pi,K}_{0,\overrightarrow{0}}])^2]^\frac{1}{4}}
\times \left[ 1 + P[\delta_{\pi,K}] \left( \frac{a\overrightarrow{p}}{\pi} \right)^2     \right].
\end{equation}

Like equation \ref{eq:dispersion_energy}, we include an additional discretization parameter $\delta_{\pi,K}$ whose prior $P[\delta_{\pi,K}] = 0 \pm 1$.  Similarly, we also use a separate parameter $\delta_\text{osc}$ to construct priors the $n=o$ oscillating kaon amplitudes only, as there is no zero twist $n=0$ oscillating pion state. 

For all $n \neq 0$ amplitudes, we have no physics-informed set prior determination method.  Additionally, we cannot use effective amplitude plots as a basis of prior determination.  Luckily, we have found in test fitting that the posterior central values of most $n\neq0$ amplitudes have similar magnitudes to their $n = 0$ counterparts of matching mass, twist, and meson type.  [UNFINISHED SECTION, need to double check]


\section{Three-point correlation function priors}
In the following subsections, I describe how I determine a full set of three-point amplitude priors, which include ground states, oscillating states, and higher order states.  A full list of such priors are given in table \ref{tab:3pt_priors}.  For a given three-point current correlator, its fit is proportional to the summed permutations of the amplitude terms $J_{ij}^{kl}$, where $i,j \in \{0,1,...,N_{max} - 1\}$, and $k,l \in \{n,o\}$.  Here $n$ denotes a non-oscillating state (rather than some order of exponential) and $o$ denotes an oscillating state.  Equation \ref{eq:3pt_correlator_fit} gives the full fit equation for $H\rightarrow\pi$.  A three point correlator for $H_s\rightarrow K$ is fit similarly.   
\begin{multline} \label{eq:3pt_correlator_fit}
    C_3^{\pi,H}(t,T) = 
    \sum_{i,j=0}^{N_\text{exp}-1} \Bigl[A^{\pi,n}_i {J_{ij}^{nn}} A^{H,n}_j e^{-E^{\pi,n}_it} e^{-E^{H,n}_j(T-t)} 
     -(-1)^{(T-t)} A^{\pi,n}_i J_{ij}^{no} A^{H,o}_j e^{-E^{\pi,n}_it} e^{-E^{H,o}_j(T-t)} \\
    -(-1)^t A^{\pi,o}_i J_{ij}^{on} A^{H,n}_j e^{-E^{\pi,o}_it} e^{-E^{H,n}_j(T-t)} 
     +(-1)^T A^{\pi,o}_i J_{ij}^{oo} A^{H,o}_j e^{-E^{\pi,o}_it} e^{-E^{H,o}_j(T-t)}\Bigr].
\end{multline}
The $J_{nn}^{00}$ term, or ground state amplitude, in this equation most important fit parameter for this work, as it is this parameter that is used in form factor construction.  In our fitting approach however, the $J_{nn}^{00}$ term is fit along side all other two and three point function related terms simultaneously.  This simultaneous fit captures, in principle, all correlations among these various fit parameters.  It is for this reason that we ensure our priors for two point function amplitudes and energies are appropriately determined: they \textit{can} have an effect on the fit posterior of the $J_{00}^{nn}$ term.

\subsection{Determining non-oscillating ground state three-point amplitude priors}\label{sec:Amp_effs_3pt}
In a similar fashion to the determination of ground state two-point amplitudes, we can use effective three-point amplitude plots to determine widths and central values for their respective priors.  To be more specific, effective three-point amplitude plots allow us estimate the prior for the $J_{00}^{nn}$ term (where $n$ denotes a \textit{non}-oscillating state, and zero denotes a state of exponential order $n=0$) in the equation \ref{eq:3pt_correlator_fit}.  The tag for this specific amplitude component is labeled \verb|Vnn0| in table \ref{tab:3pt_priors} and in the code packages I use in this work.  For a given three-point correlator, we can construct its effective amplitude from the following equation:
\begin{equation}\label{eq:3ptVeff}
    J_{\text{eff}}(t,T) = C_3(t,T)\frac{\exp\left[{M_{\text{eff}}^{\pi,K}t +M_{\text{eff}}^{B_{(s)}}(T-t)}\right]}{A_{\text{eff}}^{\pi,K} \times A_{\text{eff}}^{B_{(s)}}}.
\end{equation}

For three-point amplitudes, we cannot use the same methods of prior determination that we used for determining two-point amplitudes.  There is no expected dispersion relation we expect three-point amplitudes to follow.  In fact, the $q^2$ dependence of these three-point amplitudes is a behavior that emerges from the data alone, rather than by any strict prior enforcement.  Furthermore, as seen in equation \ref{eq:3pt_correlator_fit}, the value of a three-point correlation function for a given $t,T$ has contributions from combinations of various oscillating and non-oscillating amplitudes, even in the case where $i,j = 0$.  Additionally, These oscillating amplitude components can have either positive or negative sign.  The implication then is that even in cases where an effective three-point amplitude appears to achieve a steady plateau in some $t$ range, it cannot be assumed that the $J_{00}^{nn}$ amplitude is the only contributing state to that steady plateau.  While in correlator fit testing it was often the case that the fit posterior for a given $J_{00}^{nn}$ was consistent with some $J_{\text{eff}}(t,T)$ plateau, it was not universally true.  For these reasons we cannot justify as narrow prior widths for well resolved ground state thee-point as for similarly well resolved two-point amplitudes.

With no dispersion relation on which to base non-zero-twist three-point amplitude priors, and a desire to not unduly influence the $q^2$ dependence their corresponding fit posteriors, for much of this project we adopted a near-maximally conservative prior determination methodology.  For a given current component $J \in \{S,V,X,T\}$, on a given ensemble, we opted to use a single prior central value and uncertainty for all heavy-quark mass and twist combinations.  This central value and uncertainty was determined by visually looking at a plot (like figure \textbf{INSERT REF TO PLOT}) which included the effective amplitudes of all mass and twist combinations for a given current component. Such a plot might include twenty effective three-point amplitude plots, from which a visual grouping of plateau behavior might be resolved with varying degrees of clarity.  A central value would be determined from some apparent mean of these plateauing effective amplitudes, and the uncertainty would be determined such that both the upper and lower bounds of the plateau grouping were encompassed by the resultant prior bounds.  

This prior determination methodology was utilized through most this research project and in many acceptable correlator fits.  This methodology however is somewhat flawed when we consider a number of factors.  First, this method of prior determination does not include \textit{any} consideration of $q^2$ dependency for three-point amplitudes.  While this choice is preferable to over-constraining any $q^2$ dependency, better fit results might be achieved by having \textit{some} $q^2$ dependency built into our priors.  Second, this method somewhat misconstrues the function of a prior's uncertainty as encoded in a \verb|gvar|.  By choosing to encode a prior's central value and uncertainty in a \verb|gvar|, we encode that this central value is the mean of some normal distribution whose standard deviation is equal to the prior's uncertainty.  This is no revelation, but it means that the mindset, where any value within a prior's uncertainty range is equally preferable and any value outside of it is discouraged, is not coherent with the gaussian structure of the prior as encoded by a \verb|gvar|.  

This issue becomes more problematic when we consider how, as shown by figure \textbf{INSERT PLOT REFERENCE}, the average uncertainty or noise of high twist (low $q^2$) effective amplitudes is greater than that of low twist (high $q^2$) effective amplitudes.  Figure \textbf{PLOT REFERENCE} then also shows the general trend that effective amplitudes plateaus tend to decrease in value as twist increases. These effects combined mean that, as displayed in figure \textbf{PLOT REFERENCE}, for a given current component and heavy-quark mass, effective amplitude plateaus are larger and more stable a low twists, but are smaller and less stable at high twists.  Applying a gaussian prior, whose central value is a rough mean of this whole range of amplitudes, can create an undesirable fit outcome. In effect, the gaussian prior structure favors fit posteriors closer to the given mean or central value.  The stable plateau of a zero or low twist effective amplitude might be so well resolved that this pull towards the prior's central value has little effect in the fitter's calculation of that amplitude's fit posterior.  Conversely, at the other end of the prior uncertainty bounds, the noisy and often unstable plateaus of the corresponding high twist amplitudes will be more influenced by this pull towards the mean.  In other words, the more indeterminable a fit parameter is from the data alone, the more influence the matching parameter's prior has on the final fit posterior.  This phenomena is intended by the constructed of our augmented $\chi^2$ procedure.  Taking this all into consideration, we really ought to have different priors for the three-point amplitudes of different twist but matching current components.  We stress matching current component here because, for a given current component, the effective amplitudes plots vary in plateau position and uncertainty much more as twist is varied rather than as heavy-quark mass is varied.  

With the desire to have twist-specific priors for three-point amplitudes, we still do not want to venture into the territory of unfairly constraining the expected $q^2$ dependency.  Therefore while we determine central values by visually identifying where an effective three-point amplitude plateaus, we simply assign an uncertainty at least equal to the associated central value.  In other words, all ground state three-point amplitude priors have at least $100\%$ uncertainty.  This method is consistent with not overly constraining the expected $q^2$ dependent behavior, and avoids the "pull towards the mean" problem of unstable high twist amplitudes given the same prior as stable low twist amplitudes.  We stress "at least" here because, in some cases, an effective amplitude plot was so noisy or unstable that we felt $100\%$ uncertainty could still be too constraining.  Such cases are clearly indicated in tables \ref{tab:3pt_f5_priors}-\ref{tab:3pt_uf5_priors}.


% Figure \ref{fig:3pt_amps_fine} gives a sample \BtoPi\space scalar current effective amplitudes plot, which covers all mass and twist combinations. From plots such as this it is straightforward to visually (manually) estimate $J_{00}^{nn}$ priors.  For this research, we choose to use one prior to apply to all mass, twist, and source-sink separation $T$ combinations for a single ground state three point amplitude. This choice necessitates priors with relatively large uncertainties in order to comfortably cover the effective amplitude space spanned by all variable combinations.  In particular, a $J_{00}^{nn}$ prior's broad uncertainty comes mostly from our choice of imposing that it applies all twist options.   

% In theory, we \textit{could} use different priors to apply to different twist options (different momentum transfers, different $q^2$ in other words). From effective amplitude plots like figure \ref{fig:3pt_amps_fine} it might even appear reasonable to do so; it is possible to distinguish between five different effective amplitude plateaus for the five twist options.  However, its important to remind ourselves that the ultimate aim of this work is to calculate the form factors of \BstoPiK\space as a function of $q^2$.  A three point amplitude's twist dependence \textit{is} the basis of a form factors $q^2$ functionality\footnote{In another journal I am likely to detail the steps of this basis in more detail.}.  For this reason, we wish not to overly constrain the fitting of ground state three point amplitudes in relation to their momentum dependence.  A three point amplitude's momentum transfer dependence should be self evident from the fit results, rather than enforced \textit{a priori}.  With that said, we still ensure that these conservative prior uncertainties avoid violating the 10:1 prior to posterior rule of thumb before.  Table \ref{tab:3pt_priors} gives the current working values of these manually determined $J_{nn}^{00}$ priors.



\subsection{Determining oscillating and excited state three-point amplitude priors}
Unfortunately, determining the magnitudes of the oscillating and non-ground state three-point amplitudes cannot be simply achieved by observing effective amplitude plots.  Furthermore, for every one $J_{00}^{nn}$ state, there are \textit{many} more oscillating and higher order exponential states $J_{ij}^{kl}$ where $\{i,j\}\neq \{0,0\}$ and or $k,l \neq n$.  In fact, if $N_\text{exp} = 4$, for every one $J_{00}^{nn}$ state in a fit, there are 63 other states which contain oscillating components and or higher order exponentials.  This gives us 64 total fit parameters for every combination of current insertion, heavy quark mass, and daughter meson momentum each. Determining what the priors for these 63 non-ground states ought to be then poses an obvious challenge.  There is no algorithm or effective amplitude plot-based determination we can use;  we have no \textit{a priori} physics knowledge, other than knowing they can be positive or negative.  Thankfully, that sign indifference means we can assign a value of zero to the central value of all these non-ground states' priors.  

This still leaves us with the challenge of determining and assigning prior uncertainties for thousands of fit parameters.  Following from the similar work of \cite{Will_technical}, we can first try grouping non-ground state priors into two categories.  Per ensemble, for a given current component $J$, we can assign a single prior to all states which contain any higher order exponentials $J_{ij\neq00}^{kl}$.  This method leaves us only with the lowest lying oscillating states $J_{00}^{kl\neq nn}$, to which we can then assign a separate prior.  We can refer to these two groupings as $J_{\neq0}$ and $J_\mathrm{osc}$ as shorthand, the priors for which we can label $P[J_{\neq0}]$ and $P[J_\mathrm{osc}]$ respectively.   

With no prior physics knowledge to set the uncertainties of $P[J_{\neq0}]$ and $P[J_\mathrm{osc}]$, we used the highest uncertainty of a matching $P[J_{00}^{nn}]$ as an initial guess.  We then utilized the same $\log (\mathrm{GBF})$ optimization method discussed in section \ref{sec:prior_determination_methodology} to refine these guesses into final working values, which are listed in tables \ref{tab:3pt_pesky_priors_Hpi, tab:3pt_pesky_priors_HsK}.


% These "pesky" priors are denoted as \verb|V0| and \verb|Vn| in table \ref{tab:3pt_priors} and in the code packages I use in this work.  These terms respectively match $J_{00}^{kl}$ for $k$ and or $l \neq n$, and $J_{ij}^{kl}$ for any $i,j \neq 0$.  Alternatively in words, $V0$ denotes all lowest order state to lowest order state three-point amplitudes that contain oscillating components, and $Vn$ denotes all three-point amplitudes containing higher order components, oscillating or non-oscillating.

 % Therefor we use the a similar GBF optimization strategy as described before.  We do however still need an initial set of \verb|V0| and \verb|Vn| priors to begin GBF testing. Considering that these amplitudes can all in theory take negative values, it is natural to set the central values of these pesky priors as $0.0$.  As for their uncertainties, we assign their initial values as equal to their respective ground state non-oscillating (non-pesky) counterpart.  From this baseline, we proceed with GBF optimization testing, aborting the procedure when we come close to violating the 10:1 prior to posterior uncertainty rule.  In testing we find that the uncertainty of the GBF optimized \verb|V0| prior tends to (but not always) be at least two times larger than the uncertainty on its matching \verb|Vn| prior.  The fit posteriors to which these priors correspond to are not carried forward in the following form factor calculations, so it might be tempting to not give too much care to the priors.  However, in the domain of fitting correlation functions, our fitter does not unfairly prioritize any $J_{ij}^{kl}$ over another.  The non-oscillating ground state $J_{00}^{kl}$ is only one $J_{ij}^{kl}$ of many in our fit function (equation \ref{eq:3pt_correlator_fit}).

In test fitting, even after GBF optimizing the priors for oscillating and higher order three point amplitudes, we found two consistent outlier cases where fit posteriors still differed from their respective priors by two or more standard deviations.  These outlier cases occurred irrespective of of mass, twist, or current component.  The first of these cases were the $J_{00}^{on}$ states.  The second of these cases were the states $J_{10}^{kl}$ for all $k,l \in \{n,o\}$.  This suggested that our initial non-ground state prior grouping method could be improved.  We therefore introduced an ensemble specific, but otherwise global, multiplicative factor $\delta_\mathrm{ens}$ which effectively multiplies or \textit{widens} the uncertainty on these special case.  We then underwent the same GBF optimization for this special prior widener for each ensemble $\delta_\mathrm{ens}$, yielding values that span roughly from two to five across the five ensembles, and can also be found in tables \ref{tab:3pt_pesky_priors_Hpi, tab:3pt_pesky_priors_HsK}. This entire process is somewhat "in the weeds" for standard correlator fitting prior selection.  And while implementing these $\delta_\mathrm{ens}$ terms does not appear to have any measurable impact on the ground state amplitudes we carry forward into our form factor calculations, it does nevertheless accomplish two things.  First, it does make our correlator fitting routine slightly more conservative, that is, the affected priors are wider and less constraining to the fit.  Second, the implemented change does meet the $\Delta \log(\mathrm{GBF}) > 3$ threshold wherein we can state the change is significant. 


\section{Correlator matrices and fit partitioning}
\subsection{The need for fit partitioning}
For the \BstoPiK \space project, I have access to five different ensembles of lattice correlator data, which I will refer to fine, superfine, ultrafine. fine-physical, and superfine-physical.  
For each ensemble, there are (at least for our purposes) effectively four different dimensions over which I can preform a fit.  

The first of these is the heavy quark mass of the $B_{(s)}$ meson.  For each ensemble, there are four different heavy quark masses which scale from the charm quark up to about 80\%  the mass of the physical bottom quark.

The second fit variable is the momentum transfer to the daughter meson, or the \textit{twist}.  There is an inverse relation between the twist (daughter meson momentum) and the energy imparted to the current insertion of the three-point interaction\footnote{See figure \ref{fig:3pt_drawing} for a diagram of the three point interaction.}, or $q^2$.  For example, a twist of zero corresponds to a stationary mother meson decaying into a stationary daughter meson, where the mass difference is entirely imparted to the current insertion: maximising $q^2$.  Except for the fine-physical ensemble with six twist options, all ensembles have five twist options.


The third variable over which we can fit is $T$, or "big T".  This is the time slice $t = T$ on the lattice at which the mother meson  propagator is inserted\footnote{Counter-intuitively, the daughter meson propagator is inserted at time slice $t = 0$.  Fortunately, QCD is symmetrical under time reversal.}.  Across all five ensembles, there are four $T$ options, which roughly span between 15\% to 22\% of the temporal lattice length, plus or minus a few percent depending on the chosen ensemble\footnote{The temporal region of interest for our fit lies between $t = 0$ and $t = T$.  The result is that most of each correlator data line corresponds to a current insertion that occurs at time $ t > T$, or after the mother meson propagator is inserted.}. 

The final variable to fit over is the choice of correlator components.  This includes which three point current components for which to fit (scalar, temporal vector, spacial vector, and tensor), mother meson spin-taste constructions\footnote{to preserve various symmetries in the QCD interaction, mother meson propagators with different quantum numbers are used on the lattice to match the quantum numbers of the different current insertions.}, and the strange spectator quark counterparts of the components mentioned above.  This variable can effectively be though of as the choice of fitting over which two and three-point correlator components.

I note these four fitting dimensions to highlight an important factor: for each ensemble, the data across all four of these dimensions are correlated to \textit{some} extent.  These correlations can be represented in a two-dimensional \textit{correlation matrix}, which grows with the square of the number of variables.  The inverse of this correlation matrix is then used as a parameter in the calculation of the $\chi^2$ for a  given fit.  For this reason, a hypothetical \textit{best} fit for each ensemble would consider the whole variable range of every dimension.  The problem which arises from fitting over all four dimensions is that the power/time needed to invert the correlation matrix often exceeds the available computational resources.  

One possible solution to this problem is to \textit{not} fit over all four of these dimensions simultaneously. If I, for example, chose an ensemble and fit over each heavy quark mass independently, I can produce four sets of fit results that cover the entire variable range.  However, this solution is only viable if the correlations among the matching three point amplitudes of different masses are relatively small.  In section \ref{sec:fit_sep_by_quark_mass}, I investigate each of the five ensembles so determine the relative size of the correlations between the different heavy quark masses.  Meanwhile in section \ref{sec:fit_sep_by_curr_type}, I instead experiment with separating my fitting procedure by different three-point current correlators (and their respective mother meson correlators of appropriate spin-taste structure).  


\subsection{Fit separation by heavy quark mass}\label{sec:fit_sep_by_quark_mass}
In the following sections I document my findings on the mass correlation matrices for \BstoPiK \space.  In section \ref{sec:general_findings} a sample mass correlation matrix shown in figure \ref{fig:CorrMtrx_F}.  All other sample mass correlation matrices are to be found in this journal's appendix.  I would like to note that where I talk about relatively larger or smaller correlations among current components, I am always referring to the non-trivial off-diagonal correlations.

\subsubsection{General Findings \label{sec:general_findings}}
Across all five ensembles, there are a few common observations worth noting.  First, the four current components split into two inter-correlating pairs.  The scalar current components tend to correlate more with the temporal vector current components and vice versa, while the spacial vector current components tend to correlate more with the tensor current components.  I observe this same pairing when evaluating the behavior of three-point effective amplitude plots, where I call these the \textit{real} and \textit{imaginary} components respectively.  This current component pairing is mirrored in the available correlator data: there is no zero-twist ($q^2_{max}$) correlator data for the spacial vector and tensor current cases\footnote{While zero-twist correlator data exists for the temporal vector component, they cannot be easily used to calculate form factors, as shown in equation 5 in \cite{Parrott_2023_techincal}}.  This is (partially) why we can see the "checkerboard" pattern of higher correlations among the real current component pair, and the solid square patterns of higher correlations among the imaginary current component pair.

Looking deeper into the checkerboard patterns of the real current component correlations, we can see higher correlations among components of matching mass, and matching twist.  Meanwhile, real current components of differing masses and/or differing twists have much smaller, almost zero-valued correlations.  Even between a real and an imaginary current component, those of matching twists have correlations greater than those of differing twists.  However, this relation does not hold for matching and differing masses between real and imaginary current components.

Finally, I note that the correlations among scalar current components of matching masses or matching twists tend to be the greatest subset of correlations withing one matrix.

\subsubsection{Ensemble specific findings}
Looking at these sample mass correlation matrices more broadly, the ensembles with the largest off-diagonal correlations are, in order of largest to smallest: Fine, Fine-physical, Superfine, Ultrafine, Superfine-physical.  Among this set of sample correlation matrices, the highest degree of correlations occurs in the Fine-physical \BtoPi\space sample matrix (\ref{fig:CorrMtrx_Fp}) between the the most massive and second most massive zero-twist scalar current components, a correlation equal to $0.46$.  Additionally, all the zero-twist scalar current components in this matrix correlate with each other at values ranging from $0.39$ up to $0.46$.  There are similarly high correlations among the zero-twist scalar current components for the Fine ensemble \BtoPi\space matrix (figure \ref{fig:CorrMtrx_F}): $0.3$ to $0.4$.  

Alternatively, the maximum-twist scalar current components have the highest correlations between each other in the \BstoK\space matrices on the Fine (figure \ref{fig:CorrMtrx_Fs}) and Fine-physical ensembles (figure \ref{fig:CorrMtrx_Fps}): ranging from $0.37$ to $0.42$, and $0.27$ to $0.33$ respectively.

\subsection{Fit separation by current components}\label{sec:fit_sep_by_curr_type}
As mentioned in section \ref{sec:general_findings}, the four components of the current insertion split into two inter-correlating pairs: the scalar and temporal vector components which compose the \textit{real} pair, and the spacial vector and tensor components which compose the \textit{imaginary} pair.  The correlations between any collection real components and any collection of imaginary component seem to be relatively smaller than the heavy quark mass correlations discussed in the previous section.  This suggests that perhaps the better axis upon which to separate the fitting procedure is component type rather than heavy quark mass.  In other words, fewer (relatively) high correlations might be lost if I preform two separate fits for each ensemble: one for the real current component pair, and one for the imaginary component pair.

By preforming some smaller fits over single current component ranges, it became increasingly clear that separating fitting by heavy quark mass was less optimal than by current component.  Figures \ref{fig:Corr_Mtrx_Fp_scalar_comp} and \ref{fig:Corr_Mtrx_Fp_xvector_comp} demonstrates this idea more clearly.  I will note that absolute correlation size is, in part, a function of the size of the correlation matrix.  In other words, for a fit over a smaller number of elements, the correlations among that small number of elements will be larger than if those same elements were a fit alongside additional elements. 

With that being said, its still visibly apparent that the overall correlations between elements of opposite pair types (real or imaginary) are smaller than the correlations between elements of the same pair type by different in heavy quark mass.  This observation is more readily apparent in the course ensembles I use.


\section{Partitioned fitting method}\label{sec:partitioned_fitting_technique}
\subsection{Weighted average of daughter meson energies}
Having settled on a correlator fitting method where, for a given meson decay, the \textit{real} and \textit{imaginary} halves of the correlator data are fit separately, we are left with the question of how to properly address the daughter meson energy fit posteriors we obtain from either half of the fit.  As shown explicitly in equation \ref{eq:spin-taste_3pt_function_construction}, the same pion (or kaon) two-point correlator data are fit in both halves of our fitting regime.  This reality presents the obvious question: how should we manage separate fit results for the same correlator data?

Ideally, we want to take an average of the two matching fit posteriors such we preserve statistical correlations.  Not only should this average account for the high correlation between the two matching posteriors, but this new average should still preserve its correlations with the all other fit posteriors in either half of the fit.  Thankfully, the \verb|lsqfit| package contains a \textit{weighted average} function that preserves these correlations, and accounts for the relative uncertainties of either \verb|gvar| in the weighted average.  This new \verb|gvar| is then a weighted average of the matching posteriors from both halves of the fit, which we can easily save for future use in form factor calculations.  Each new, weighted average \verb|gvar| however is statistical least-squares fit in its own right, and carries its own fit statistics ($\chi^2$, degrees of freedom, quality factor $Q$). 

In correlator fit testing, it became clear that in the minority of cases, some weighted average \verb|gvar|s had $\chi^2/\mathrm{d.o.f.} > 1$.  Such instances corresponded to the matching fit posterior from both halves of the partitioned fit not being within one standard deviation of the other.  In these cases we can follow the guidelines given in the PDG \cite{ParticleDataGroup:2024cfk} on how to scale the uncertainty of a weighted average to account for its status as a poor fit.  To do so for some mean value $g$, we can multiply its standard deviation $\sigma_g$ by some scaling factor $\lambda$, where
\begin{equation}\label{eq:chi2_scaling}
    \lambda = \sqrt{\chi^2/(\mathrm{d.o.f.}-1)}. \quad \textbf{DOUBLE CHECK THIS}
\end{equation}
Given that we are managing \verb|gvar|s, we can use an indirect method to scale uncertainty without removing the existing correlations.  Suppose we label the final, scaled \verb|gvar| we want to save for future calculations as $z \pm\sigma_z$.  We then wish to create some uncorrelated \verb|gvar| $y\pm\sigma_y$ which, when multiplied by the original unscaled weighted average \verb|gvar| $x\pm\sigma_x$ gives a final scaled \verb|gvar| $z \pm\sigma_z$ where $z = x$, $\sigma_z = \lambda \times\sigma_x$.  Lets begin with the standard error propagation formula for this scenario:
\begin{equation}\label{eq:error_prop}
    \left(\frac{\sigma_z}{z}\right)^2 = \left(\frac{\sigma_x}{x}\right)^2 + \left(\frac{\sigma_y}{y}\right)^2.
\end{equation}
For our desired case of scaling the uncertainty of the original weighted average $\sigma_x$ by $\lambda$, we set $z = x$, $\sigma_z = \lambda \sigma_x$, and $y = 1$.  We then have
\begin{equation}\label{eq:error_prop_pt2}
\begin{split}
    \left(\frac{\lambda\sigma_x}{x}\right)^2 &= \left(\frac{\sigma_x}{x}\right)^2 + \sigma_y^2,\\
    \sigma_y^2 &=   \left(\frac{\lambda\sigma_x}{x}\right)^2 - \left(\frac{\sigma_x}{x}\right)^2,\\
               &= \frac{\sigma_x^2(\lambda^2-1)}{x^2},\\
    \sigma_y   &= \frac{\sigma_x\sqrt{\lambda^2-1}}{x}.
\end{split}
\end{equation}
With this formulation established, we could easily alter our \verb|python| code such that scaled weighted average values for daughter meson energies and amplitudes are saved for future form factor calculations.