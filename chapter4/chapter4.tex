\chapter{Fitting Correlation Functions}\label{cha:fitting_correlation_functions}
This chapter will likely be one of the more substantial sections in my thesis.  In chapters \ref{DstopiK} and \ref{cha:BstopiK} I will discuss my findings for $B_{(s)}\rightarrow \pi(K)$ and then $D_{(s)}\rightarrow \pi(K)$.  These four different meson decays are all built on the same lattice data sets to which I have access.  As a result, the same methodology in fitting various correlation functions is largely shared among these four meson decays.  The section outline for this chapter will likely go as follows:
\begin{itemize}
    \item \textbf{Lattice units}.  In much of the subsequent math to follow in this chapter, we work in lattice units for masses and energies.  Therefore masses $M$ and energies $E$ in equations are actually $aM$ and $aE$.  The instances in which this will not be true are where we talk explicitly in physical units of GeV.  
    \item \textbf{Correlation Function Calculation}. Here is where I intend to establish from first principles the characteristics of two and three-point correlation functions.  It will be necessary to discuss phases in our staggered quarks, spin-taste structure, creation and annihilation operators, and current insertions.  This will be followed by discussion of sources and sinks on the lattice, random domain walls, and twisted boundary conditions.
    \item \textbf{Correlator Fit functions}.  Here we get into the realm of Lattice QCD which which I am most familiar as of writing.  I can introduce the classic two and three-point correlator fit functions, equations \ref{eq:2ptcorrfit} and \ref{eq:3ptcorrfit}.  It would also be conducive to include some sample log(correlator) plots.
\end{itemize}

\begin{itemize}
    \item \textbf{Bayesian Fitting} This might be a shorter section, but detailing our Bayesian fitting approach likely warrants its own section.  This section will briefly summarize why a correlator fitting methodology which involves Bayesian fitting, as outlined in \textit{constrained curve fitting} does not compromise the statistics of our work.  Naturally this discussion introduces the idea of gaussian priors, which neatly leads us to the next section.

     \item \textbf{SVD Cuts}.  This might be a relatively smaller section, but it will be important to discuss the reality of having more data than samples, the covariance matrix and eigenvalues.
    \item \textbf{Determining Goodness of Fit}.  Here I will describe tools we use to determine the goodness of any particular fit.  These include evaluating $\chi^2/\text{d.o.f.}$, and log(GBF).  I would likely include a section on SVD and prior noise as well, with reference to the artificial lowering of $\chi^2/\text{d.o.f.}$ that comes from wide priors and SVD cuts.
\end{itemize}

\section{The fundamentals of correlator fitting}\label{sec:fundamentals_of_correlator_fitting}
\subsection{Fit equations}
We now have two different descriptions of correlation function on the lattice.  The first of these defines the correlation function of some meson $\mathcal{M}$ in terms of energies $E_i^\mathcal{M}$ and amplitudes $A^\mathcal{M}_i$.  The second description defined the correlation function in terms quark propogators, and phases which relate to a meson's spin-taste structure.  The form our generated correlator data takes is the latter of these two, and by equating the two definitions of the correlation function we can fit this correlator data to the fit form prescribed by the former.
Equations \ref{eq:2ptcorrfit} and \ref{eq:3ptcorrfit} give the expected fit forms for our two and three-point correlation functions.  We can first explore the fit form for the two-point correlation function:
\begin{equation}\label{eq:2ptcorrfit}
    C_2^\mathcal{M}(t)=\sum^{N_\mathrm{exp}-1}_{i=0}\big(|A_i^{\mathcal{M},\mathrm{n}}|^2(e^{-E_i^{\mathcal{M},\mathrm{n}}t}+e^{-E_i^{\mathcal{M},\mathrm{n}}(N_t-t)}) -(-1)^{t}|A_i^{\mathcal{M},\mathrm{o}}|^2(e^{-E_i^{\mathcal{M},\mathrm{o}}t}+e^{-E_i^{\mathcal{M},\mathrm{o}}(N_t-t)})\big),
\end{equation}
Here we can discuss the various terms that appear in these fit forms.  First we again note that the lattice operators used to generate the correlator data couple not only to some ground state of energy $E^\mathcal{M}_{i=0}$, but to an infinite tower of excited energy states $E_i^\mathcal{M}$ for all $i>0$.  In our correlator fitting we limit the number of exponentials states to some number $N_\mathrm{exp}$, hence the sum over some number of states $i \in [0,N_\mathrm{exp})$.  In section \ref{sec:Nmax_and_tmin} we discuss how we determine an appropriate value for $N_\mathrm{exp}$.  Likewise we must also consider the periodic boundary conditions we impose on the lattice.  Such conditions effectively imply the presence of some meson at the temporal end of the lattice $N_t$ moving "backwards" towards the opposite end $t_0$. This phenomena accounts for the $(N_t-t)$ factors which appear in equation \ref{eq:2ptcorrfit}.  Finally, as discussed in section \ref{sec:opposite_parity}, meson states with opposite parity manifest in the correlator as states whose phases oscillate in time $t/a$ (or simply $t$ in equations \ref{eq:2ptcorrfit} and \ref{eq:3ptcorrfit} where the factor of the unit lattice length $a$ is implied).  Energies and amplitudes that correspond to such oscillating states have superscripts $o$, while negative parity, non oscillating states have superscripts $n$.  A sample two-point correlator plot is given in figure \ref{fig:2pt_logcorr}, where we plot the $\log$ of the correlator to show how the Monte Carlo correlator data is modeled well by the exponential form of the model function: equation \ref{eq:2ptcorrfit}. 

\begin{figure}
    \centering
    \includegraphics[width=0.8\linewidth]{chapter4/pdfs/custom2pt_logcorr.pdf}
    \caption{Sample $H$ and $\pi$ $\log(C^\mathcal{M}_2)$ plot on the f5 ensemble.  Error bars are too small to be visible. Both the heavy meson $H$ and light meson $\pi$ have Goldstone pseudoscalar spin taste structure $\gamma_5\otimes\xi_5$.  The heavy quark in $H$ has mass $am_h=0.675$, and the light meson has twist $\theta = 1.282$, which is related to momentum via equation \ref{eq:twist_angle}.} 
    \label{fig:2pt_logcorr}
\end{figure}

We can now go on to discuss the fit form of the three-point correlation function:
\begin{equation}\label{eq:3ptcorrfit}
  \begin{split}
    C^{\mathcal{M}_1,\mathcal{M}_2}_3(t,T)&=\sum^{N_\mathrm{exp}-1}_{i,j=0}\big(A_i^{\mathcal{M}_1,\mathrm{n}}J_{ij}^{\mathrm{nn}}A_j^{\mathcal{M}_2,\mathrm{n}}e^{-E_i^{\mathcal{M}_1,\mathrm{n}}t}e^{-E_j^{\mathcal{M}_2,\mathrm{n}}(T-t)}\\
    &-(-1)^{(T-t)}A_i^{\mathcal{M}_1,\mathrm{n}}J_{ij}^{\mathrm{no}}A_j^{\mathcal{M}_2,\mathrm{o}}e^{-E_i^{\mathcal{M}_1,\mathrm{n}}t}e^{-E_j^{\mathcal{M}_2,\mathrm{o}}(T-t)}\\
    &-(-1)^{t}A_i^{\mathcal{M}_1,\mathrm{o}}J_{ij}^{\mathrm{on}}A_j^{\mathcal{M}_2,\mathrm{n}}e^{-E_i^{\mathcal{M}_1,\mathrm{o}}t}e^{-E_j^{\mathcal{M}_2,\mathrm{n}}(T-t)}\\
    &+(-1)^{T}A_i^{\mathcal{M}_1,\mathrm{o}}J_{ij}^{\mathrm{oo}}A_j^{\mathcal{M}_2,\mathrm{o}}e^{-E_i^{\mathcal{M}_1,\mathrm{o}}t}e^{-E_j^{\mathcal{M}_2,\mathrm{o}}(T-t)}\big).
  \end{split}
\end{equation}
Here we have two two-point function which propagate either lattice locations $t_0$ or $T$, where \textit{width} $T$ is the source-sink separation of the meson propogators.   We insert the appropriate current $J$ at all $t$, which correspond to the amplitude of the overlap between the two meson propogators.  For a small enough width $T$ and large enough temporal lattice length $N_t$, the signal from the "backwards" moving meson (a result of the periodic boundary conditions) can be ignored, and therefore no factors of $(N_t-t)$ appear in the fit form.   Figure \ref{fig:3pt_logcorr} shows a sample three-point log-correlator plot, which likewise shows how the Monte Carlo correlator data is modeled well by the exponential form of the model function: equation \ref{eq:3ptcorrfit}.  These fit forms act as the model correlation functions $C_2^\mathrm{fit}$ and $C_3^\mathrm{fit}$ which we utilize in our augmented least squares fitting approach, which we discuss in the following subsection.  
\begin{figure}
    \centering
    \includegraphics[width=0.8\linewidth]{chapter4/pdfs/custom3pt_logcorr.pdf}
    \caption{Sample $H\rightarrow\pi$, $\log(C^S_3)$ plot on the f5 ensemble.  Error bars are too small to be visible. These three point correlator correspond to a scalar current insertions, a heavy quark mass $am_h=0.675$, and a light meson with twist $\theta = 1.282$, which is related to momentum via equation \ref{eq:twist_angle}.  Four correlators with different values of source-sink separation $T$ are shown.} 
    \label{fig:3pt_logcorr}
\end{figure}


\subsection{Bayesian fitting}\label{sec:bayesian_fitting}
For this research, we utilize a Bayesian fitting approach \cite{Carlin1997} to mitigate a fundamental hurdle to fitting correlation functions: often we seek to fit for more parameters (the various energies and amplitudes that appear in equations \ref{eq:2ptcorrfit} and \ref{eq:3ptcorrfit}) than we have correlator data.  In an non-augmented least squares fitting approach, this can result in unbounded uncertainties on fit parameters.  In the Bayesian approach, we effectively increase the sum of data we fit with the inclusion of \textit{a priori} estimates of all fit parameters.  Including these estimates, or \textit{priors}, as data in the fit ensures that there are always more data than parameters.  This then avoids the possibility that the fitter's determinations of those parameters, the fit \textit{posteriors}, have infinite uncertainties.  In this procedure we augment the normal $\chi^2$ equation one might expect to minimize in least squares fitting such that $\chi^2 \rightarrow \chi^2_\mathrm{aug} = \chi^2 + \chi_{\text{prior}}^2$.  The priors we insert into correlator fitting are gaussian variables, such that for some parameter $X$ we write that it's prior $P[X] = \Tilde{X} \pm \Tilde{\sigma}_X$, where $ \Tilde{X}$ is the prior's central value and $\Tilde{\sigma}_X$ is it's gaussian standard deviation (or uncertainty, which we colloquially use interchangeably).  As mentioned previously, we utilize the \textit{gvar} package \cite{gvar_package} throughout this work to manage these priors.  Among other things, this package allows the user to store and operate with gaussian random variables, and preserves the statistical correlations between such variables.

The application of this approach to correlator fitting specifically is well demonstrated in \cite{bayesian}, whose notation we also follow from.  Considering a generic two-point correlator fit, we can then write the equation for $\chi^2_\mathrm{aug}$ as
\begin{equation}\label{eq:least_squares_invert_covariance}
    \chi_\mathrm{aug}^2(E_i,A_i) = \sum_{t,t'}\Delta C_2(t)\space\sigma^{-2}_{t,t'}\space\Delta C_2(t') 
    + \sum_i  \frac{(E_i - \Tilde{E}_i)^2}{(\Tilde{\sigma}_{E_i})^2} + \sum_i \frac{(A_i - \Tilde{A}_i)^2}{(\Tilde{\sigma}_{A_i})^2}.
\end{equation}
The first term in equation \ref{eq:least_squares_invert_covariance} is standard in least squares fitting.  Here 
\begin{equation}\label{eq:corr_average-model}
    \Delta C_2(t) = \overline{C_2^\mathrm{data}(t)} - C_2^\mathrm{model}(t,E_i,A_i),
\end{equation}
Where $\overline{C_2^\mathrm{data}(t)}$ is the averaged value of a Monte Carlo correlation function (the data) at $t$, and $C_2^\mathrm{model}(t,E_i,A_i)$ is the value of the model correlation function as defined by \ref{eq:2ptcorrfit} at $t$.  It is the parameters in the model correlation function that are iterated upon by the fitter such that $\chi^2$ is minimized.  We also find the first term of equation \ref{eq:least_squares_invert_covariance} the normalized covariance matrix\footnote{Another name for the normalized covariance matrix is the \textit{correlation} matrix.  For the sake of not adding another "correlate"-containing phrase to this work (correlation function, correlator data, statistical correlations) we prefer the more clear, if somewhat clunky, "normalized covariance matrix".} $\sigma_{t,t'}$, which we define as 
\begin{equation}
    \sigma^2_{t,t'} = \overline{C_2^\mathrm{data}(t)C_2^\mathrm{data}(t')} -  \overline{C_2^\mathrm{data}(t)}\text{ } \overline{C_2^\mathrm{data}(t')}.
\end{equation}
The second and third terms in our definition of $\chi^2_\mathrm{aug}$ make up $\chi^2_\mathrm{prior}$.  From this construction we can predict how the assignment of priors might affect the minimization of $\chi^2_\mathrm{aug}$.  For smaller, or more \textit{narrow} prior uncertainties, the fitter is more strongly penalized when the fit parameter differs from its matching priors central value.  Meanwhile, if a prior's uncertainty is relatively large, then the minimization of $\chi^2_\mathrm{aug}$ will be relatively less sensitive to that prior's central value.  Sections \ref{sec:prior_determination_methodology},\ref{sec:2pt_priors}, and \ref{sec:3pt_priors} discuss our methods of prior determination extensively.

All of our correlator fitting is handled using the \textit{corrfitter} package \cite{peter_lepage_2021_5733391}.  This package integrates our Bayesian fitting approach to correlator fitting, along side the \textit{lsqfit} package \cite{peter_lepage_2024_12690493}.  Both these packages utilize the \textit{gvar} package such that fit parameter uncertainties and correlations are  accounted for during the fitting process.


% These priors augment the standard $\chi^2$ fitting procedure (equations \ref{eq:augmented_chi2} and \ref{eq:chi2_prior_expression})  The inclusion of priors in my fitting methods follows the methodology first outlined in \cite{bayesian}, and I provide an effective summation and justification here.  In the general fitting process of a meson two-point correlator, we attempt to extract a set of energies $E_n$ and amplitudes $A_n$ from the correlator (or set of correlators)  $C_2(t)$, which we can write generally as
% \begin{equation}\label{eq:meson_correlator}
%     C_2(t) = \sum_{n=0}^\infty A_n e^{-E_nt}.
% \end{equation}
% In this common scenario, we are troubled with managing an infinite stacked tower of $n^{th}$ order energies and amplitudes with correlators of finite temporal length (which for this project range between $t_{\text{max}/a} = 96$ to $t_{\text{max}/a} = 192$).  

% In \cite{bayesian} the host of problems that result from this reality are mitigated by the introduction of \textit{constrained curve fitting}.  In this method, we introduce an \textit{a priori} estimate for the mean and standard deviation\footnote{The default prior distribution choice is a Gaussian. In section 3.1 of \cite{bayesian} the arbitrary nature of this distribution choice is detailed more fully.} of a given $E_n$ or $A_n$ in the fitting procedure, a set of which we can collectively call \textit{the priors} of a given fit.  This method, among other things, enforces realistic physical behavior of higher order energies and amplitudes.  Following from equations 5 and 6 in \cite{bayesian}, we augment the normal $\chi^2$ fitting procedure:
% \begin{equation}\label{eq:augmented_chi2}
%     \chi^2 \rightarrow \chi^2_\text{aug} = \chi^2 + \chi_{\text{prior}}^2.
% \end{equation}
% We can denote the prior on a given $E_n$ or $A_n$ as $\Tilde{E_n} \pm \Tilde{\sigma}_{E_n}$ or $\Tilde{A}_n \pm \Tilde{\sigma}_{A_n}$ respectively.  This defines $\chi_{\text{prior}}^2$ as
% \begin{equation}\label{eq:chi2_prior_expression}
%     \chi_{\text{prior}}^2 = .
% \end{equation}

% In the generation of correlators (from $N_f = 2 + 1 + 1$ MILC-HISQ gluon fields \cite{MILC_2010, MILC_2012}), we impose that the amplitudes $A_n$ are "well behaved" in that they have non-infinite values.  This imposes a behavior of rapid decay among higher order states along the temporal lattice length.  This allows us to begin fitting at some $t=t_\text{min}$, where the contributions to the correlator data at $t \geq t_\text{min}$ are dominated by the first few and lowest ordered states.  Adjusting $t_\text{min}$ comes with the trade-off of different systematic errors, but most importantly, it reduces the effective number of priors for each state (a value of $N_\mathrm{exp} = 4$ has been common throughout this project).

% In tables \ref{tab:2pt_priors} and \ref{tab:3pt_priors}, I give a (working and mostly comprehensive) list of priors that go into my fitting script.  In the following sections I go into greater detail explaining what these priors are, how they are decided upon, and which priors can be determined with computational methods. 


% Equation where we need to invert the covariance matrix of correlated data, following notation from \cite{Yoon2013}:
% \begin{equation}\label{eq:covariance_matrix}
%     S_{ij} = \frac{1}{N_\mathrm{cfg}-1}\sum_{n=1}^{N_\mathrm{cfg}}\left[C_i(n) - \bar{C_i}\right]\left[C_j(n) - \bar{C_j}\right]
% \end{equation}
% We can then define a normalized sample covariance matrix $\sigma_{ij}$ where
% \begin{equation}\label{eq:norm_sample_cov}
%     \sigma_{ij} = \frac{1}{N_\mathrm{cfg}}S_{ij}
%\end{equation}

\section{Partitioning the correlator fit}\label{sec:partitioned_fitting_technique}
\subsection{Motivation}
While the computational resources we have access to for this research are generous, they are not sufficient to fit the sum of our correlator data simultaneously.  Increasing the size of correlator data included in a fit will non-linearly increase the computing time of a fit.  For this work, attempts to simply fit all $H\rightarrow\pi$ (or $H_s\rightarrow K$) correlator data simultaneously exceeded the allowed clock-time per job submitted to the Cambridge/DiRAC computer cluster we use.

Perhaps the most significant contributor to correlator fit computation time is the inversion of the data covariance matrix (recall the term $\sigma^2_{t,t'}$ in equation \ref{eq:least_squares_invert_covariance}), especially when data is both highly precise and correlated \cite{PhysRevD.43.196, DRUMMOND1993271, KILCUP1994350, PhysRevD.51.3745}. There are a number of techniques we can utilize that could reduce the resources needed to calculate $\sigma^{-2}_{t,t'}$ \cite{Yoon2013}, the testing and implementation of which we discuss here and in the following sections. 

Perhaps the most direct approach we can take to reduce the computation time needed to invert the covariance matrix is to separate the fitting routine into distinct parts.  This is to say, to not fit all Monte Carlo correlator data simultaneously,  but to partition the fit such that the absolute size of the covariance matrix is reduced.  This approach is analogous to the diagonal approximation discussed in \cite{PhysRevD.49.2616, Yoon2013}, where off diagonal elements in the covariance matrix are set equal to zero.  In either case, the time needed to invert the covariance matrix is reduced.  Naturally, a partitioned fit approach will not account for any statistical correlations among separately fit data.  If the statistical correlations among separately fit data are relatively large, we could jeopardize the validity of our fit results; the $\chi^2$s of the separate fits would be underestimated.  Therefore if we can determine which groupings of correlator data have relatively small statistical correlations with other grouping of correlator data, we can partition our fit routine along the lines that differentiate those groupings.

From this point onward, it will be useful to talk about statistical correlations among the fit parameters (energies and amplitudes) rather than our Monte Carlo correlator data.  These parameters, as determined by the augmented least squares fit, acquire statistical correlations from the correlator data input into equation \ref{eq:least_squares_invert_covariance}.  They are therefor useful proxies we can use to investigate covariance matrices.  It should then be useful to catalog the dimensions over which we can categorize our fit parameters.  

Per meson decay (2: $H\rightarrow\pi$, $H_s\rightarrow K$), per ensemble (5: f5, fphys, sf5, sfphys, uf5), we can categorize fit parameters by heavy quark mass $am_h$, and meson or current insertion spin-taste structure (equation \ref{eq:correlator_spin-taste}).  Our three-point correlator data does also vary by source-sink separation $T$.  We do not however consider source-sink separation as a dimension though which to partition our fit routine.  This is because, for a constant heavy quark mass, twist, and current insertion, varying $T$ does not change which parameters $J_{ij}^{kl}(am_h,\theta)$ are determined by the fitter.  


% For the \BstoPiK \space project, I have access to five different ensembles of lattice correlator data, which I will refer to fine, superfine, ultrafine. fine-physical, and superfine-physical.  
% For each ensemble, there are (at least for our purposes) effectively four different dimensions over which I can preform a fit.  

% The first of these is the heavy quark mass of the $B_{(s)}$ meson.  For each ensemble, there are four different heavy quark masses which scale from the charm quark up to about 80\%  the mass of the physical bottom quark.

% The second fit variable is the momentum transfer to the daughter meson, or the \textit{twist}.  There is an inverse relation between the twist (daughter meson momentum) and the energy imparted to the current insertion of the three-point interaction\footnote{See figure \ref{fig:3pt_drawing} for a diagram of the three point interaction.}, or $q^2$.  For example, a twist of zero corresponds to a stationary mother meson decaying into a stationary daughter meson, where the mass difference is entirely imparted to the current insertion: maximizing $q^2$.  Except for the fine-physical ensemble with six twist options, all ensembles have five twist options.


% The third variable over which we can fit is $T$, or "big T".  This is the time slice $t = T$ on the lattice at which the mother meson  propagator is inserted. Counter-intuitively, the daughter meson propagator is inserted at time slice $t = 0$.  Fortunately, QCD is symmetrical under time reversal..  

%Across all five ensembles, there are four $T$ options, which roughly span between 15\% to 22\% of the temporal lattice length, plus or minus a few percent depending on the chosen ensemble\footnote{The temporal region of interest for our fit lies between $t = 0$ and $t = T$.  The result is that most of each correlator data line corresponds to a current insertion that occurs at time $ t > T$, or after the mother meson propagator is inserted.}. 

%The final variable to fit over is the choice of correlator components.  This includes which three point current components for which to fit (scalar, temporal vector, spacial vector, and tensor), mother meson spin-taste constructions\footnote{to preserve various symmetries in the QCD interaction, mother meson propagators with different quantum numbers are used on the lattice to match the quantum numbers of the different current insertions.}, and the strange spectator quark counterparts of the components mentioned above.  This variable can effectively be though of as the choice of fitting over which two and three-point correlator components.

\subsection{Analyzing normalized covariance matrices}\label{sec:Normalized_covariance_matrices}
% I note these four fitting dimensions to highlight an important factor: for each ensemble, the data across all four of these dimensions are correlated to \textit{some} extent.  These correlations can be represented in a two-dimensional \textit{correlation matrix}, which grows with the square of the number of variables.  The inverse of this correlation matrix is then used as a parameter in the calculation of the $\chi^2$ for a  given fit.  For this reason, a hypothetical \textit{best} fit for each ensemble would consider the whole variable range of every dimension.  The problem which arises from fitting over all four dimensions is that the power/time needed to invert the correlation matrix often exceeds the available computational resources.  

% One possible solution to this problem is to \textit{not} fit over all four of these dimensions simultaneously. If I, for example, chose an ensemble and fit over each heavy quark mass independently, I can produce four sets of fit results that cover the entire variable range.  However, this solution is only viable if the correlations among the matching three point amplitudes of different masses are relatively small.  In section \ref{sec:fit_sep_by_quark_mass}, I investigate each of the five ensembles so determine the relative size of the correlations between the different heavy quark masses.  Meanwhile in section \ref{sec:fit_sep_by_curr_type}, I instead experiment with separating my fitting procedure by different three-point current correlators (and their respective mother meson correlators of appropriate spin-taste structure).  
Let us now explore some test fits and their resultant normalized covariance matrices to determine where it might be best to partition our fitting routine.  Consider the matrix presented in figure \ref{fig:mtrx_masstest}.  Here we present a normalized covariance matrix as a stylized heatmap, which makes it easier to visually spot clusters of inter-correlating parameters.  In this plot we see the relative correlations among heavy mesons energies $E_H$ of two different spin-taste structures, and the matching\footnote{Here we say "matching" to refer to their shared presence in three-point correlator construction (equation \ref{eq:spin-taste_3pt_function_construction}).} scalar and temporal vector current insertion amplitudes $S$, $V^0$.  While the twist angle $\theta =0.0$ is constant for all parameters, the mass of the heavy quark is varied.  
\begin{figure}
    \centering
    \includegraphics[width=0.9\linewidth]{chapter4/pdfs/masstest_custom_mtrx.pdf}
    \caption{Sample normalized covariance matrix of correlator fit posteriors.  Results are shown for the $H\rightarrow\pi$ meson decay on the f5 ensemble.  The sample fit covers a subset of available correlator data.  Fit parameters correspond to a range heavy quark masses $am_h$, light meson twist $\theta = 0.0$, and source sink separation (in lattice units) $T = 24$.  Parameters shown are either heavy meson energies $E_{H}$ or current insertion amplitudes $J\in\{S,V\}$, whose spin-taste structures are presented in equation \ref{eq:correlator_spin-taste}.}
    \label{fig:mtrx_masstest}
\end{figure}

We can immediately spot the pattern of $3\times3$ squares which correspond to the three different heavy quark masses present in the test fit.  The four $3\times3$ blocks along the diagonal show the extent of relative correlations among parameters of matching meson spin-taste or current insertion. These statistical correlations are relatively higher than elsewhere in the sample matrix.  Should heavy quark mass be the candidate dimension along which to partitioning the fit function, these statistical correlations would be lost, and we might invalidate the results of our least squares fit.  In this test fit and in others across all ensembles and both meson decays, we see this consistent relationship among fit parameters of shared twist and spin-taste structure, but varying heavy quark mass.  This observation rules out partitioning our final fit routine along the lines of heavy quark mass.

Next we can examine twist angle as a possible means by which to partition our fit function.  Like before, we can perform a small test fit where, while keeping heavy quark mass constant, we vary twist angle and observe the clusters of statistical correlations.  Figure \ref{fig:mtrx_twisttest} presents the normalized covariance matrix of a test fit that follows these constraints.  In this matrix we can see that, other than the diagonal of self-correlations equal to $1.0$, the highest relative correlations are among groups of parameters of matching spin-taste but differing twist angle: the three $3\times3$ blocks along the diagonal of the matrix.  Like before, this pattern suggests that twist angle is also not the best dimension along which to partition our fit routine.  Across many test fits for all ensembles and meson decays, we observe a similar pattern where parameters of matching spin-taste but varying twist angles have higher relative statistical correlations than those among parameters of matching twist but differing spin-taste.
\begin{figure}
    \centering
    \includegraphics[width=0.7\linewidth]{chapter4/pdfs/twisttest_custom_mtrx.pdf}
    \caption{Sample normalized covariance matrix of correlator fit posteriors.  Results are shown for the $H\rightarrow\pi$ meson decay on the f5 ensemble.  The sample fit covers a subset of available correlator data.  Fit parameters correspond to a heavy quark mass $am_h = 0.450$, a range of light meson twist $\theta$, and source sink separation (in lattice units) $T = 24$.  Parameters shown are either light meson energies $E_{\pi}$ or current insertion amplitudes $J\in\{S,T\}$, whose spin-taste structures are presented in equation \ref{eq:correlator_spin-taste}.}
    \label{fig:mtrx_twisttest}
\end{figure}

These first two matrices point us in the direction of the final option we have to partition the fit function: to separate the fit along the lines of spin-taste construction.  Figure \ref{fig:mtrx_Hlpi_F_E} shows a normalized covariance matrix for a test fit of this type.  The parameters shown all correspond to the same heavy quark mass (except for the pion) and the same twist angle (except for the heavy mesons which all have zero three-momentum). The first observation we can make is that the presented pion energy correlates relatively little with any other parameter.  Furthermore, regardless of spin taste construction, the heavy meson energies have relatively high correlations among themselves.  These same heavy meson energies are then also highly correlated with the amplitudes of the scalar and temporal vector currents, but not the spatial vector or tensor currents.  Counterintuitively, the energies of the heavy mesons $H''$ and $H'''$ are not highly correlated with the amplitudes $V^1$ and $T$, even though they share the same three-point function.  Figure \ref{fig:mtrx_Hlpi_F_Amp} presents a normalized covariance matrix from the same test fit, but presents the amplitudes of heavy and light mesons instead of their energies.  We can observe the same patterns of correlations here match those of the matrix in figure \ref{fig:mtrx_Hlpi_F_E}. 
\begin{figure}
    \centering
    \includegraphics[width=0.7\linewidth]{chapter4/pdfs/Hlpi_F_E_custom_mtrx.pdf}
    \caption{Sample normalized covariance matrix of correlator fit posteriors.  Results are shown for the $H\rightarrow\pi$ meson decay on the f5 ensemble.  The sample fit covers a subset of available correlator data.  Fit parameters correspond to a heavy quark mass $am_h=0.450$, light meson twist $\theta = 0.4281$, and source sink separation (in lattice units) $T = 24$.  Parameters shown are either meson energies $E_\mathcal{M}$ or current insertion amplitudes $J\in\{S,V,T\}$, whose spin-taste structures are presented in equation \ref{eq:correlator_spin-taste}.  Labels for heavy meson energies also indicate which current insertion $J$ is present in its appropriate three point function (equation \ref{eq:spin-taste_3pt_function_construction}).}
    \label{fig:mtrx_Hlpi_F_E}
\end{figure}
\begin{figure}
    \centering
    \includegraphics[width=0.7\linewidth]{chapter4/pdfs/Hlpi_F_Amp_custom_mtrx.pdf}
    \caption{Sample normalized covariance matrix of correlator fit posteriors.  Results are shown for the $H\rightarrow\pi$ meson decay on the f5 ensemble.  The sample fit covers a subset of available correlator data.  Fit parameters correspond to a heavy quark mass $am_h=0.450$, light meson twist $\theta = 0.4281$, and source sink separation (in lattice units) $T = 24$.  Parameters shown are either meson amplitudes $A_\mathcal{M}$ or current insertion amplitudes $J\in\{S,V,T\}$, whose spin-taste structures are presented in equation \ref{eq:correlator_spin-taste}.  Labels for heavy meson amplitudes also indicate which current insertion $J$ is present in its appropriate three point function (equation \ref{eq:spin-taste_3pt_function_construction}).}
    \label{fig:mtrx_Hlpi_F_Amp}
\end{figure}

We find similar patterns of correlations among all $H\rightarrow\pi$ ensembles: figure \ref{fig:mtrx_Hlpi_UF_E} shows a similar normalized covariance matrix for the uf5 ensemble instead of the f5 ensemble. We do however we observe a slight difference in examining a similar set of parameters from a test fit for $H_s\rightarrow K$.  Consider the matrix in figure \ref{fig:mtrx_HsK_F_E}: here the light meson energy does have somewhat higher statistical correlations with the heavy meson energies.  Additionally, the heavy meson energies are all highly correlated with each other, but not the amplitudes of the scalar or temporal vector currents they way they are in the case for $H\rightarrow\pi$.  We observe that these observations for either $H\rightarrow\pi$ and $H_s\rightarrow K$ are consistent across all ensembles.

\begin{figure}
    \centering
    \includegraphics[width=0.7\linewidth]{chapter4/pdfs/HsK_F_E_custom_mtrx.pdf}
    \caption{Sample normalized covariance matrix of correlator fit posteriors.  Results are shown for the $H_s\rightarrow K$ meson decay on the f5 ensemble.  The sample fit covers a subset of available correlator data.  Fit parameters correspond to a heavy quark mass $am_h=0.450$, light meson twist $\theta = 0.4281$, and source sink separation (in lattice units) $T = 24$.  Parameters shown are either meson energies $E_\mathcal{M}$ or current insertion amplitudes $J\in\{S_s,V_s,T_s\}$, whose spin-taste structures are presented in equation \ref{eq:correlator_spin-taste}.  Labels for heavy meson energies also indicate which current insertion $J$ is present in its appropriate three point function (equation \ref{eq:spin-taste_3pt_function_construction}).}
    \label{fig:mtrx_HsK_F_E}
\end{figure}
\begin{figure}
    \centering
    \includegraphics[width=0.7\linewidth]{chapter4/pdfs/Hlpi_UF_E_custom_mtrx.pdf}
    \caption{Sample normalized covariance matrix of correlator fit posteriors.  Results are shown for the $H\rightarrow\pi$ meson decay on the uf5 ensemble.  The sample fit covers a subset of available correlator data.  Fit parameters correspond to a heavy quark mass $am_h=0.194$, light meson twist $\theta = 0.706$, and source sink separation (in lattice units) $T = 44$.  Parameters shown are either meson energies $E_\mathcal{M}$ or current insertion amplitudes $J\in\{S,V,T\}$, whose spin-taste structures are presented in equation \ref{eq:correlator_spin-taste}.  Labels for heavy meson energies also indicate which current insertion $J$ is present in its appropriate three point function (equation \ref{eq:spin-taste_3pt_function_construction}).}
    \label{fig:mtrx_Hlpi_UF_E}
\end{figure}

Focusing now on three-point amplitudes, we find that across all ensembles and both meson decays, the amplitudes of scalar and temporal vector currents are relatively more correlated with each other than either of them are with the amplitudes of spatial vector or tensor currents.  The reverse of this also appears to be true.  Figure \ref{fig:mtrx_UF_currtest} gives a clear example of this phenomenon.  In this figure we see varying heavy quark mass and twist angle options, as well as the four different current insertions; the $S$ and $V^0$ amplitudes correlate more with each other than with $V^1$ or $T$, and vice versa. 
\begin{figure}
    \centering
    \includegraphics[width=\linewidth]{chapter4/pdfs/UF_currtest_custom_mtrx.pdf}
    \caption{Sample normalized covariance matrix of correlator fit posteriors.  Results are shown for the $H\rightarrow\pi$ meson decay on the uf5 ensemble.  The sample fit covers a subset of available correlator data.  Fit parameters correspond to a range of heavy quark masses and light meson twists, and source sink separation (in lattice units) $T = 44$.  Parameters shown are current insertion amplitudes $J\in\{S,V,T\}$, whose spin-taste structures are presented in equation \ref{eq:correlator_spin-taste}.}
    \label{fig:mtrx_UF_currtest}
\end{figure}

 The consistency of this behavior across our entire data set then forms the basis of our finalized correlator fit routine.  The majority of the Monte Carlo correlator data we are fitting correspond to three-point functions, which we can separate into two similarly sized categories: $J\in\{S,V^0\}$, and $J\in\{V^1,T\}$.  Out fit method is then as follows: we fit all two-point correlators with the $S$, $V^0$ three-point correlators.  Then, in the same fitting script, we fit all two-point correlators again with all $V^1$, $T$ three point correlators.  Having fit all two point correlators twice, we then calculate a weighted average of all doubly-fit parameters, the exact methodology of which we describe in the following section.  Finally, all fit outputs are saved using a \textit{gvar} package function which preserves the statistical correlations between those fit outputs.
 
% \subsection{Fit separation by heavy quark mass}\label{sec:fit_sep_by_quark_mass}
% In the following sections I document my findings on the mass correlation matrices for \BstoPiK \space.  In section \ref{sec:general_findings} a sample mass correlation matrix shown in figure \ref{fig:CorrMtrx_F}.  All other sample mass correlation matrices are to be found in this journal's appendix.  I would like to note that where I talk about relatively larger or smaller correlations among current components, I am always referring to the non-trivial off-diagonal correlations.


% \subsubsection{General Findings \label{sec:general_findings}}
% Across all five ensembles, there are a few common observations worth noting.  First, the four current components split into two inter-correlating pairs.  The scalar current components tend to correlate more with the temporal vector current components and vice versa, while the spacial vector current components tend to correlate more with the tensor current components.  I observe this same pairing when evaluating the behavior of three-point effective amplitude plots, where I call these the \textit{real} and \textit{imaginary} components respectively.  This current component pairing is mirrored in the available correlator data: there is no zero-twist ($q^2_{max}$) correlator data for the spacial vector and tensor current cases\footnote{While zero-twist correlator data exists for the temporal vector component, they cannot be easily used to calculate form factors, as shown in equation 5 in \cite{Parrott_2023_techincal}}.  This is (partially) why we can see the "checkerboard" pattern of higher correlations among the real current component pair, and the solid square patterns of higher correlations among the imaginary current component pair.

% Looking deeper into the checkerboard patterns of the real current component correlations, we can see higher correlations among components of matching mass, and matching twist.  Meanwhile, real current components of differing masses and/or differing twists have much smaller, almost zero-valued correlations.  Even between a real and an imaginary current component, those of matching twists have correlations greater than those of differing twists.  However, this relation does not hold for matching and differing masses between real and imaginary current components.
% \subsection{Fit separation by current components}\label{sec:fit_sep_by_curr_type}
% As mentioned in section \ref{sec:general_findings}, the four components of the current insertion split into two inter-correlating pairs: the scalar and temporal vector components which compose the \textit{real} pair, and the spacial vector and tensor components which compose the \textit{imaginary} pair.  The correlations between any collection real components and any collection of imaginary component seem to be relatively smaller than the heavy quark mass correlations discussed in the previous section.  This suggests that perhaps the better axis upon which to separate the fitting procedure is component type rather than heavy quark mass.  In other words, fewer (relatively) high correlations might be lost if I preform two separate fits for each ensemble: one for the real current component pair, and one for the imaginary component pair.

% By preforming some smaller fits over single current component ranges, it became increasingly clear that separating fitting by heavy quark mass was less optimal than by current component.  Figures \ref{fig:Corr_Mtrx_Fp_scalar_comp} and \ref{fig:Corr_Mtrx_Fp_xvector_comp} demonstrates this idea more clearly.  I will note that absolute correlation size is, in part, a function of the size of the correlation matrix.  In other words, for a fit over a smaller number of elements, the correlations among that small number of elements will be larger than if those same elements were a fit alongside additional elements. 

% With that being said, its still visibly apparent that the overall correlations between elements of opposite pair types (real or imaginary) are smaller than the correlations between elements of the same pair type by different in heavy quark mass.  This observation is more readily apparent in the course ensembles I use.

\subsection{Weighted average of fit parameters}\ref{sec:wavg_fit_params}
In the previous section, we settled on a correlator fitting method where, for a given meson decay and ensemble, the $J\in\{S,V^0\}$ and $J\in\{V^1,T\}$ halves of the three-point correlator data are fit separately.  To ensure that the statistical correlations between the two-point data and either half of the three point data are conserved, the two-point data is fit twice: once with either half of the three-point data.  We are then left with the question of how to properly address the meson energy and amplitude fit posteriors copies we obtain from both halves of the fit.  As shown explicitly in equation \ref{eq:spin-taste_3pt_function_construction}, the same meson two-point correlator data are fit in both halves of our fitting regime.  This reality presents the obvious question: how should we manage separate fit results for the same correlator data?

Ideally, we want to take an average of the two matching fit posteriors such we preserve statistical correlations.  Not only should this average account for the high correlation between the two matching posteriors, but this new average should still preserve its correlations with the all other fit posteriors in either half of the fit.  Thankfully, the \textit{lsqfit} package contains a \textit{weighted average} function that preserves these correlations, and accounts for the relative uncertainties of either \verb|gvar| object (the gaussian random variable \textit{object}, not the \textit{gvar} package itself) in the weighted average.  This new \verb|gvar| is then a weighted average of the matching posteriors from both halves of the fit, which we can easily save for future use in form factor calculations.  Each new, weighted average \verb|gvar| however is a statistical least-squares fit in its own right, and carries its own fit statistics ($\chi^2$, degrees of freedom, quality factor $Q$). 

In correlator fit testing, it became clear that in the minority of cases, some weighted average \verb|gvar|s had $\chi^2/\mathrm{d.o.f.} > 1$.  Such instances corresponded to the matching fit posterior from both halves of the partitioned fit not being within one standard deviation of the other.  In these cases we can follow the guidelines given in the PDG \cite{ParticleDataGroup:2024cfk} on how to scale the uncertainty of a weighted average to account for its status as a poor fit.  To do so for some mean value $g$, we can multiply its standard deviation $\sigma_g$ by some scaling factor $\lambda$, where
\begin{equation}\label{eq:chi2_scaling}
    \lambda = \sqrt{\chi^2/(\mathrm{d.o.f.}-1)}.
\end{equation}
Given that we are managing \verb|gvar|s, we want to use an indirect method to scale uncertainty without directly assigning a new uncertainty, and thereby removing the existing correlations.  Suppose we label the final, scaled \verb|gvar| we want to save for future calculations as $z \pm\sigma_z$.  We then wish to create some uncorrelated \verb|gvar| $y\pm\sigma_y$ which, when multiplied by the original unscaled weighted average \verb|gvar| $x\pm\sigma_x$ gives a final scaled \verb|gvar| $z \pm\sigma_z$ where $z = x$, $\sigma_z = \lambda \times\sigma_x$.  Let us begin with the standard error propagation formula for this scenario:
\begin{equation}\label{eq:error_prop}
    \left(\frac{\sigma_z}{z}\right)^2 = \left(\frac{\sigma_x}{x}\right)^2 + \left(\frac{\sigma_y}{y}\right)^2.
\end{equation}
For our desired case of scaling the uncertainty of the original weighted average $\sigma_x$ by $\lambda$, we set $z = x$, $\sigma_z = \lambda \sigma_x$, and $y = 1$.  We then have
\begin{equation}\label{eq:error_prop_pt2}
\begin{split}
    \left(\frac{\lambda\sigma_x}{x}\right)^2 &= \left(\frac{\sigma_x}{x}\right)^2 + \sigma_y^2,\\
    \sigma_y^2 &=   \left(\frac{\lambda\sigma_x}{x}\right)^2 - \left(\frac{\sigma_x}{x}\right)^2\\
               &= \frac{\sigma_x^2(\lambda^2-1)}{x^2},\\
    \sigma_y   &= \frac{\sigma_x\sqrt{\lambda^2-1}}{x}.
\end{split}
\end{equation}
With this formulation established, we can easily alter the fitting script to calculate weighted averages of meson energies and amplitudes that respect all statistical correlations in our fit.  These new averages are then saved alongside the three-point parameters, the amplitudes of the current insertions, which we then use further in the process of form factor calculation.


\subsection{Chained and marginalized fits}\label{sec:chained_and_marginalized}
There are two more methods for reducing correlator fit computation time which we explored for this research, but ultimately did not implement in our final fit routine.  These methods are \textit{chained} fits, and \textit{marginalized} fits.  Both of these methods (detailed in the documentation for the \textit{lsqfit} package \cite{peter_lepage_2024_12690493}) entail truncating or modifying the default \textit{lsqfit} routine of fitting all parameters simultaneously. Both these methods manage the correlator fitting in a somewhat more streamlined manner, as compared to the full partitioned fit routine we finally settled on.  So while we do not inclue them in our final fit, it is worth exploring what they are and describing how we came to decision to not utilize them.

In a chained fit, one chooses a subset of parameters to be fit first: meson energies and amplitudes for example.  The fit posteriors of this first fit are then used as (ideally very precise) priors for the next step in a chained fit.  In this next step, or link, a new subset of parameters (some current insertion amplitudes for example) is fit alongside those already fit in the previous link.  This process continues for a number of user designated links until all parameters are fit simultaneously, where perhaps most of those parameters are given (very) narrow priors equal to the outputs from previous links.  This method can, among other things, reduce the number of iterations the fitter takes to minimize the $\chi^2_\mathrm{aug}$ function.  Additionally, it does not remove statistical correlations among fit parameters.  In testing however it was often the case that including three-point correlator data in a fit would meaningfully affect the fitter's determination of a mesons anergy or amplitude.  Furthermore, we found that link composition and ordering significantly influenced final fit posteriors.  Finally, even with a chained fit of all correlator data on a given ensemble and  meson decay, we still hit the computation time limit.  For these reasons we did not utilize a chained fit in our finalized fitting routine.

Where a chained fit partially separates the fitting of certain parameters, a marginalized fit entails removing some parameters from the fit more directly.  Most of the parameters in our fit forms (equations \ref{eq:2ptcorrfit} and \ref{eq:3ptcorrfit}) do not appear in the equations needed to calculate lattice form factors (see chapter \ref{cha:form_factors}).  Such parameters include all higher order ($i>0$) and oscillating terms which appear in the fit forms.   In a marginalized fit we then use priors of these select parameters to model their respective parts of the fit function, which are then subtracted from the input data. This technique can be used to effectively prioritize the ground state parameters for which we have the most interest.  Selecting which higher order and or oscillating parameters to marginalize requires a study of test fits to determine which are appropriate.  We wish to avoid the case where marginalizing a certain set of non-ground state parameters significantly influences the fitter's determination of their ground state counterparts.  In test fitting we found that across all ensembles there was no consistent set of parameters which, when marginalized, did not either speed up the fitting process or significantly influence the ground state parameters.  For this reason we do not marginalize any parameters in out finalized fit technique.

\section{Prior determination methodology}\label{sec:prior_determination_methodology}
In this section we describe the general methods we use to determine the priors for the fit parameters which appear in equations \ref{eq:2ptcorrfit} and \ref{eq:3ptcorrfit}.   These methods tend to fall into two approximate categories: manual and automated.  We can manually assign some priors by visually analyzing effective mass, energy, and amplitude plots (such as figure \ref{fig:effective_mass_plots})  So long as the correlator data from which these effective plots are constructed are not too noisy, ground state behavior can be visually (which is to say, manually) differentiated from higher order and oscillating states. 

The determination of some priors from effective plots is sometimes best done algorithmically.  In such cases where a ground state plateau is distinct enough from non-ground state signal, a function can take the rolling average of adjacent and consecutive effective plot values.  The central value of the ground state energy or amplitude can be determined from where the change in this rolling average is minimized.

Irrespective of methodology, with prior determination we always aim to set conservative priors, avoiding cases where priors impose strict constraints on their matching posteriors.  In test fitting, we ensure that prior absolute uncertainties are no smaller than at least ten times their matching posteriors.  We take exception to this principle when we have additional physics information to which we expect the standard model-derived correlator data to adhere.  Sections \ref{sec:dispertion_relation} and \ref{sec:excited_state_energies} details examples of this in relation to the dispersion relation and excited state meson energies respectively.  In such cases, physics based algorithms are used to automatically assign values to these priors, and often the $10:1$ prior to posterior uncertainty rule is violated.

Finally, in the cases where we cannot use effective amplitude plots, nor physics based preconceptions to determine priors, we utilize Gaussian Bayes Factor (GBF) optimization. Much more information on the Bayes factor can be found in \cite{sivia2006data, carlin2010bayes}, but to effectively summarize its use here: prior tuning that maximizes a fit's GBF indicates that the prior neither over-constrains nor under-constrains the fit \cite{bayes_Kass_Rafferty}.  For priors indeterminable by the previously noted methods, such as oscillating excited three-point current insertion amplitudes, we opt to use this method.  A prior is GBF optimized by performing a test fit, and recording the fit's GBF, or more often $\log(\mathrm{GBF})$.  Then, the prior's mean or standard deviation is slightly adjusted.  The fit is performed once more and the subsequent change (if any) of the $\log(\mathrm{GBF})$ is recorded.  This process is repeated until a clear maximum is resolvable. The GBF optimization of a prior's uncertainty is aborted however if maximizing a fits GBF violates the 10:1 prior to posterior uncertainty principle.  In such cases, more conservative (larger standard deviation) priors are chosen over more precise, GBF-preferred priors.  As a benchmark, a change in $\log(\mathrm{GBF})$ greater than 3 is considered significant for purposes od GBF testing.


\section{Two-point correlation function priors}\label{sec:2pt_priors}
In the following subsections, I discuss the priors on meson amplitudes and energies.  These priors are used in equation \ref{eq:chi2_prior_expression} to fit two point correlator functions (equation \ref{eq:meson_correlator}).  These priors cover the whole range of two-point correlator related fit parameters. This includes the ground state and higher order exponential state energies and amplitudes of: mother mesons with varying masses and spin-taste structures, daughter mesons of varying momentum, and the daughter meson oscillating state counterparts.

%For each set two-point correlators that correspond to a $B_{(s)}$ of mass $am_b$, there are four correlator subsets for the four spin-taste copies 

\subsection{Meson rest masses}\label{sec:meson_rest_masses}
The first set of priors we can discuss are the priors for the rest masses of the heavy and light mesons.  By construction, the heavy mother mesons in this work are in the rest frame of the lattice, and so their energies are equivalent to their masses.  The lighter daughter mesons however are simulated at varying momenta, whose priors we discuss in \ref{sec:dispertion_relation}.  In this formulation, a daughter meson of zero 3-momentum corresponds to a $q^2_{\text{max}}$ current insertion; the daughter meson remains at rest in the lattice frame and its energy equals its rest mass.

For a two-point correlator $C_2(t)$, we can plot its effective mass $aM_{\text{eff}}(t)$ via equation \ref{eq:2ptam}.  For a well behaved\footnote{Which is to say, the higher order exponential states rapidly decay.} two-point correlation function, it is possible to extract an estimate for its rest mass from its effective mass plot.  
\begin{equation} \label{eq:2ptam}
        aM_{\text{eff}}(t) = \frac{1}{2}\cosh^{-1}\left(\frac{C_2(t-2)+C_2(t+2)}{2C_2(t)}\right). 
\end{equation}

Figure \ref{fig:aMeff_sample} shows sample effective mass plots for $H\rightarrow \pi$.  From this figure we can see that the effective mass function follow a trend where, after $t_{\text{min}}$, the higher order states decay quickly.  Then, the data stabilizes with relatively small uncertainty and minimal central value fluctuations.  This is the signal for the ground state rest mass.  Then, the uncertainty and the size of central value fluctuations grows towards the middle of the lattice length.

This consistent behavior (a steady region of relatively noiseless ground state energy/rest mass signal) in the effective mass plot permits the use of an automated prior determination method.  We utilize a simple custom \verb|Python| function which takes a rolling average over sets of 4 adjacent effective mass values (that is to say: at 4 consecutive lattice time slices $t/a$) and finds where this rolling average changes the least.  The central value of the rest mass's prior is then assigned to be equal to change-minimized, rolling averaged $aM_{\text{eff}}(t)$.  Figure \ref{fig:effective_mass_plots} gives a visual example on how this minimum change in the rolling average is related to the effective mass of a two-point correlation function.  

\begin{figure}
    \centering
    \includegraphics[width=0.9\linewidth]{chapter4/pdfs/CustomEs.pdf}
    \caption{Sample pion effective mass plots on the f5 and uf5 ensembles.  The effective mass $aM\mathrm{eff}(t)$, which is calculated from equation \ref{eq:2ptMeff} and its respective two point correlation function $C^\pi_2$, is shown in blue. The rolling average of the effective mass R.Avg$(t)$ is shown in dashed green. Where the change in this rolling average is minimized $\Delta_\mathrm{min}$ of R.Avg$(t)$ is shown in dotted red. Consecutive red bands show regions of $\pm5\%$, out to $\pm30\%$, of $\Delta_\mathrm{min}$ of R.Avg$(t)$.}
    \label{fig:effective_mass_plots}
\end{figure}

Using this method we can observe effective mass plots and assign fractional uncertainties to our priors for ground state meson rest masses.  Tables \ref{tab:Hpimeson_mass_and_amp_priors} and \ref{tab:Hpimeson_mass_and_amp_priors} gives the complete list of the fractional uncertainness we use in our correlator fits, which mostly range between $0.05-0.10$.  Across all cases, we confirm that the standard deviations of meson mass fit posteriors are at least ten times smaller than the standard deviations of their matching fit prior.  In many cases, the prior to posterior standard deviation ratio is much greater: on the order 100:1 or greater.  This is an assuring sign that the priors we have chosen are not constricting to the fit.  It is important to note that, after spotting such large prior to posterior ratios in test fitting, we do not then adjust priors to move closer to the ratio 10:1.  Even in cases where such a change would increase the $\log(\mathrm{GBF})$ of a fit, we would be violating the justification for using priors in the first place.  We wish to maintain physics and or data backed reasons for prior choices, otherwise the priors lose their \textit{a priori} nature.

% \begin{table}[]
%     \centering
%     \caption{Priors for dispersion relation parameters}
%     \begin{tabular}{c c c c c c c}
%         \hline
%         Ensemble & $P[\epsilon^{\pi,n}]$ & $P[\epsilon^{\pi,o}]$ & $P[\mathcal{A}^\pi]$ & $P[\epsilon^{K,n}]$ & $P[\epsilon^{K,o}]$ & $P[\mathcal{A}^K]$ \\
%         \hline
%          f5     & 0.0(1.0)  & 0.0(1.0) & 0.0(1.0) & 0.0(1.0)  & 0.0(1.0) & 0.0(1.0) \\
%          fphys  & 0.0(36.0) & 0.0(1.0) & 0.0(1.0) & 0.0(50.0) & 0.0(1.0) & 0.0(1.0) \\
%          sf5    & 0.0(1.0)  & 0.0(1.0) & 0.0(1.0) & 0.0(5.0)  & 0.0(1.0) & 0.0(1.0) \\
%          sfphys & 0.0(1.0)  & 0.0(1.0) & 0.0(1.0) & 0.0(3.0)  & 0.0(1.0) & 0.0(1.0) \\
%          uf5    & 0.0(1.0)  & 0.0(1.0) & 0.0(1.0) & 0.0(1.0)  & 0.0(1.0) & 0.0(1.0) \\
%          \hline
%     \end{tabular}
%     \label{tab:dispersion_params}
% \end{table}
 
%Table \ref{tab:aMeff_values} shows the prior and posterior estimates for the same masses displayed in figure \ref{fig:aMeff_sample}.  From this table, we can see that the fit posteriors for these masses (zero-twist energies) have significantly smaller percent errors than the prior percent errors.  For this reason, I consistently use this automated method to determine priors for meson masses.

%The above method works for both the mother and daughter mesons in this study.  

\subsection{Zero-momentum oscillating meson masses}\label{sec:oscillating_mass_priors}
% As discussed in section \textbf{INSERT REF TO PARITY - OSCILALTION FOR HISQ}, the quantum number states with positive parity, $J^P = (0^+)$ for example, manifest as HISQ-derived correlation functions with oscillatory behavior.    
Having established the methodology for determining priors on our ground state, zero-twist, non-oscillating meson energies (masses), we can determine priors for their ground state, zero-twist, \textit{oscillating} counterparts.  The ground state mesons have angular momentum and parity quantum numbers of $J^P = 0^-$.  The meson states with quantum numbers $J^P = 0^+$ also couple to the correlation functions.  As discussed in section \ref{sec:opposite_parity}, our use of the HISQ action \cite{HISQ:2006rc} leads to opposite parity states manifesting as oscillating propagators on the lattice.  We therefor call the lowest energy meson state with quantum numbers $J^P = 0^+$ the \textit{lowest lying oscillating} state of that meson. These are the states which have energies and amplitudes $E_0^{\mathcal{M},o}$ and $A_0^{\mathcal{M},o}$ in equation \ref{eq:2ptcorrfit}.  We discuss the amplitudes of these states later in section \ref{sec:2ptamps_non_ground}. As noted in section \ref{sec:opposite_parity}, the zero-momentum oscillating pion does not appear in its respective two point correlator.  However an estimate for its energy will be used later in section \ref{sec:dispertion_relation} to set priors non-zero-momentum oscillating pion energies.  For this reason we still take care to determine reasonable priors for $E_0^{\pi,o}$.

Unfortunately, priors for these lowest lying oscillating states cannot be determined using the same methodology as before; we can not visually determine a lowest lying oscillating state's contribution to an effective mass plot. Instead, we can rely on the known energy difference, or splitting, between the non-oscillating state and lowest lying oscillating state given by the Particle Data Group. \cite{ParticleDataGroup:2024cfk}.  For Example, the ground state pion has a rest mass of about 140 GeV, while the lowest energy pion state with opposite parity, the $f_0(500)$ has a rest has a mass of roughly 500 GeV.  This difference equals 360 MeV, or roughly $0.7\times\Lambda_\mathrm{QCD}$.  This state however does not have the same isospin quantum number as the ground state pion, and so is not coupled to our pion propogators.  Instead we must look to the $a_0(980)$ particle, which shares the same isospin quantum number as the ground state pion.  The energy difference between those two particles is then 840 MeV, or $1.68 \times \Lambda_\mathrm{QCD}$. Knowing this energy difference allows us to set the central values of the priors for the lowest lying oscillating state energies $P[E_0^{\pi,o}] = P[E_0^{\pi,n}] + 1.68 \times \Lambda_\mathrm{QCD}$.  This works across all ensembles because we define the central value of $P[E_0^{\pi,o}]$ relative to the central value of the ground state pion on a given ensemble, and the some scalar times $\Lambda_\mathrm{QCD}$.  

Setting the uncertainty for $P[E_0^{\pi,o}]$ is not a trivial task meanwhile.  The known energy difference, or mass splitting, for the physical continuum pion and $a_0(980)$ is not necessarily equal to the mass splitting between those equivalent states on the lattice.  For finite lattice spacing, the the order $\mathcal{O}(a^2)$ taste symmetry breaking leads to non-degenerate quark masses of different tastes \cite{HISQ:2006rc, PhysRevD.77.074505}. This effect will be larger for the positive parity scalar pion state we are trying to identify in our correlator fitting.\cite{PhysRevD.111.094506, Bernard:2006gj}.  Additionally, this $a_0$ resonance is strongly coupled to certain multi-particle channels of matching quantum numbers \cite{PhysRevD.93.094506}.  These factors combined means that the lowest lying oscillating state we extract from our correlator fitting may differ from the physical $a_0$ by hundreds of MeV.  For this reason, we assign a large uncertainty to the priors $P[E_0^{\pi,o}]$ equal to $\pm (0.5\times \Lambda_\mathrm{QCD})$.

For the kaon in our $H_s\rightarrow K$ fitting, we see similar roadblocks to pinning down the lowest lying oscillating state.  Our candidate oscillating state particle with quantum numbers $I(J^p) = \frac{1}{2}(0^+)$ is the $K^*_0(700)$.  This state however is both very broad, and also strongly coupled to multi-particle states of matching quantum numbers \cite{PhysRevD.91.054008}, $\pi K$ scattering in particular \cite{PhysRevD.102.114520}.  For these reasons and the taste splitting effects mentioned previously, we can expect to that the $\frac{1}{2}(0^+)$ state our fitter will identify has an mass splitting closer to $\Lambda_\mathrm{QCD} = 500$ MeV rather than $200$ MeV.  For this reason, we set the central values of $P[E_0^{K,o}] = P[E_0^{K,n}] + \Lambda_\mathrm{QCD}$.  We then likewise assign those priors an uncertainty equal to $\pm (0.5\times \Lambda_\mathrm{QCD})$.

Thankfully, determining the priors for the masses of the lowest lying oscillating $H$ and $H_s$ states is more simple. From HQET we expect the splitting to not be significantly on the heavy quark mass, with a consistent splitting of $\mathcal{O}(\Lambda_\mathrm{QCD})$ \cite{Neubert:1994, Gandhi:2022epjc}.  Considering the discretization effects inherent to our HISQ ensembles, and $1/m_b$ corrections more specifically \cite{LANG201517}, we expect to see a mass splitting closer to 400 MeV, or $0.8\times\Lambda_\mathrm{QCD}$\cite{PhysRevD.70.054501}.  Across all ensembles an heavy quark masses we therefore give the priors central values equal to the ground state central value plus $0.8\times\Lambda_\mathrm{QCD}$.  Meanwhile, because of this more defined mass splitting, we can select a small fractional uncertainty of $0.2$ for these priors. This fractional uncertainty approaches $\Lambda_\mathrm{QCD}$ at the heavier end of our heavy quark options.

For all these oscillating meson states, we check that the standard deviations of the fit posteriors are not overly constrained by the standard deviations of their matching priors. Across all ensembles and in both $H\rightarrow\pi$ and $H_s\rightarrow K$ we find this to be the case.

\subsection{Non-zero-momentum meson energies: the dispersion relation}\label{sec:dispertion_relation}
Having established our methodology for assigning priors to zero-momentum meson energies, we can now move on to discuss how to assign priors for non-zero-momentum meson energies.  Across all ensembles we place the lattice frame in the rest frame of the heavy meson, and so we need only consider pions and kaons in this regard.  

As discussed in section \ref{sec:twisted_boundary_conditions}, we impart momentum to the light mesons via twisted boundary conditions.  We can describe a meson's 3-momentum $\vec{p}$ in terms of a twist angle $\theta$, where
\begin{equation}\label{eq:twist_angle}
    \theta = |a\vec{p}| \times \frac{N_x}{\sqrt{3}\pi},
\end{equation}
and $N_x$ is the spacial length of the lattice in lattice units.  Equation \ref{eq:2ptam} provides a template for how we \textit{could} determine the priors for these light meson energies.  In fact the equation is valid for any meson effective energy $aE_\mathrm{eff}$ of any momentum in our cur context, so in this context we can rewrite \ref{eq:2ptam} as
\begin{equation} \label{eq:2ptaE}
        aE_{\text{eff}}(t) = \frac{1}{2}\cosh^{-1}\left(\frac{C_2(t-2)+C_2(t+2)}{2C_2(t)}\right). 
\end{equation}
Figure \ref{fig:pion_effective_energies} shows the range of pion effective energies of varying twists on the ultrafine 5 ensembles. 
\begin{figure}
    \centering
    \includegraphics[width=0.9\linewidth]{chapter4/pdfs/aE_uf5_Hpi.pdf}
    \caption{The full range of pion effective energies of varying twists on the ultrafine 5 ensemble.  Light meson correlators $C^\pi_2$ of twist $\theta$ are used in equation \ref{eq:2ptaE} to calculate effective energies $aE_\mathrm{eff}(t)$.  The meson's twist is related to its lattice momentum via equation \ref{eq:twist_angle}.}
    \label{fig:pion_effective_energies}
\end{figure}

For assigning non-zero-twist meson energies however, we can employ some \textit{a priori} physics knowledge to move beyond simply observing effective energy plots.  We can instead employ an energy dispersion relation which we expect the the non-zero-twist mesons to follow, which at the continuum and in natural units is $E = \sqrt{M^2 + p^2}$.  On the lattice, we modify the traditional, continuum limit dispersion relation with a term that accounts for discretization effects.  With the HISQ action \cite{HISQ:2006rc}, the lowest order discretization effects we expect in this dispersion relation are of the order $(a\vec{p})^2$.  Using the dispersion relation we can assign the priors of non-zero-twist meson energies $P[aE^\mathcal{M}_{\vec{p}}]$ as
\begin{equation}\label{eq:dispersion_energy}
   P[aE^\mathcal{M}_{\vec{p}}] = \sqrt{P[aE^\mathcal{M}_{\vec{0}}]^2 + (a\vec{p})^2}
   \times \left[ 1 + P[\epsilon^\mathcal{M}] \left( \frac{a\vec{p}}{\pi} \right)^2     \right],
\end{equation}
where $P[aE^\mathcal{M}_{\vec{0}}]$ is the prior of the meson's energy at zero-twist (it's rest mass).  Here we also introduce the discretization parameters $\epsilon^\mathcal{M}$, which scales the relative size of discretization effects on the dispersion relation.  A given $\epsilon^\mathcal{M}$ is specific to the light meson being fit (pion or kaon in our case) on a given ensemble.  For two light mesons and five ensembles, we introduce ten new parameters into our correlator fitting routine, which therefor need priors as well.  Following the findings from comparable HISQ-action fits \cite{Bouchard_2013, Will_towardsB, Chakraborty_2021, Will_technical}, the priors for $\epsilon^{\mathcal{M}}$ were set as $P[\epsilon^\mathcal{M}] = 0\pm1$ throughout much of this project's timeline.  In various test fits for this work, the posteriors for $\epsilon^{\pi}$, $\epsilon^{K}$ fell within the bounds of these priors.  

In later stages of this project however we noticed that, on select ensembles, the fit posteriors for this discretization parameter consistently differed from its prior by one or several standard deviations.  This prompted us to apply the standard $\log(\mathrm{GBF})$ testing routine discussed in \ref{sec:prior_determination_methodology} to the prior uncertainties.  We however cut halted the $\log(\mathrm{GBF})$ testing when it would have preferred prior uncertainties smaller $\pm1.0$.  This was the case for most most $P[\epsilon^\mathcal{M}]$ uncertainties, and so the majority of dispersion parameters priors remain as $P[\epsilon^\mathcal{M}] = 0\pm1$.  The most distinct deviation from this norm is manifests in the fine physical (fphys) ensemble for both the pion and the kaon.  The $\log(\mathrm{GBF})$ preferred uncertainties for $P[\epsilon^{\pi}]$, $P[\epsilon^{K}]$ are $\pm 36.0$ and $\pm50.0$ respectively.  While these uncertainties are relatively large, in test fitting we find that the fphys ensemble fit posteriors for $P[\epsilon^{\pi}]$, $P[\epsilon^{K}]$ have large, greater than one fractional uncertainties, $6(13)$ and $3.0(9.8)$ for example.  Furthermore, such large uncertainties for $P[\epsilon^{\pi}]$, $P[\epsilon^{K}]$ on the fphys ensemble mean that the priors for high twist light meson energies $P[aE^\mathcal{M}_{\vec{p}}]$ on that ensemble also have relatively large fractional uncertainties: $0.75-1.0$ for example, following from equation \ref{eq:dispersion_energy}.  

The conclusion we come to from these observations is that, on the fphys ensemble, some combination of momentum related discretization effects greatly reduces our ability to set narrower priors for high twist pion and kaon energies.  Thankfully, the signal from these high twist meson energies is strong enough that in test fitting, despite their broad priors, the posteriors of these energies have fractional uncertainties that range from $0.01-0.04$.  For the $H_s\rightarrow K$ decay exclusively, we also found for the sf5 and sfphys ensembles, $\log(\mathrm{GBF})$ testing preferred $P[\epsilon^{K}]$ uncertainness of $\pm5.0$ and $\pm3.0$ respectively.  In test fitting, the fit posteriors for $\epsilon^{K}$ were both consistent with their priors, but also with $0.0\pm1.0$.


In addition to the psuedoscalar ground state pion and kaons, we can use the dispersion relation to assign priors for the non-zero-twist lowest lying oscillating pion and kaon energies as well.  In fact, we can use the same equation as before, equation \ref{eq:dispersion_energy}, with one minor change.  That change is to introduce additional notation (which been otherwise ignored up until this point) to differentiate between oscillating and non-oscillating energies.  This change applies to energies: $E^{\mathcal{M},n}_{\vec{p}}$, $E^{\mathcal{M},o}_{\vec{p}}$, and the discretization parameter as well: $\epsilon^{\mathcal{M},n}$,  $\epsilon^{\mathcal{M},o}$.  It is here where all the work we have done to determine the prior of $E^{\pi,o}_{\vec{0}}$ becomes useful.  Even though the zero-momentum oscillating pion state does not contribute to the pion correlator, we can still us if to determine the non-zero-momentum oscillating pion states, which do contribute to their correlators.

For various reasons, we can expect that the discretization parameter $\epsilon^{\mathcal{M},o}$ will be weakly determined in correlator fitting.  The first of these is that that the uncertainties for $P[E^{\mathcal{M},o}_{\vec{0}}]$ are already quite large: $\pm0.5\times\Lambda_{QCD}$.  This means that the contribution to the high momentum $P[E^{\mathcal{M},o}_{\vec{p}}]$ uncertainties coming from the uncertainty of $P[E^{\mathcal{M},o}_{\vec{0}}]$ will still be relatively large.  This is in contrast to the ground state light mesons with small $P[E^{\mathcal{M},o}_{\vec{0}}]$ uncertainties, where at high momentum the uncertainty assigned to $P[E^{\mathcal{M},o}_{\vec{p}}]$ via equation \ref{eq:dispersion_energy} is dominated by the uncertainty of $P[\epsilon^{\mathcal{M},n}]$.  The second reason we might expect a weak determination of $\epsilon^{\mathcal{M},o}$ is that the oscillating contribution to the two point correlator decays quickly with time, making it more difficult for the fitter to determine a precise value for $\epsilon^{\mathcal{M},o}$.  For these reasons, for both $H\rightarrow\pi$ and $H_s\rightarrow K$, across all ensembles, we set all $P[\epsilon^{\mathcal{M},0}]=0.0\pm1.0$.  We find in test fitting that fit posteriors of $P[\epsilon^{\mathcal{M},o}]$ are consistent with their priors, having central values close to zero, with standard deviations less than or near order one.

It is important to note that $\epsilon$'s use in this fitting procedure is limited only to the construction on non-zero twist meson energy priors.  In other words, $\epsilon$ does not \textit{directly} constrain a fit's posterior value for any $aE^\mathcal{M}_{0,\overrightarrow{p}}$.  In the fitting procedure, various $aE^{\pi,K}_{0,\overrightarrow{p}}$ are fit no differently than any other fit parameter (equation \ref{eq:chi2_prior_expression}). 

In the later stages of this project, we observed in test fitting consistent disagreement between the fit posteriors of non-zero-twist meson energies and their matching priors in some select cases.  In an attempt to reconcile this disagreement, we revisited our use of the dispersion relation.  Ignoring the discretization parameter contribution $1\pm\epsilon(a\vec{p}/\pi)^2$, the dispersion relation of the form $aE=\sqrt{(aM)^2+(a\vec{p})^2}$ is a truncated version of what we would expect on the lattice.  Because the light mesons are bosons, we can follow from equation 6.39 in \cite{Smit_2002} and write their full, non-truncated energy dispersion relation on the lattice as
\begin{equation}\label{eq:non_trunc_dispersion}
    aE^\mathcal{M}_{\vec{p}} = \cosh^{-1}{\left[1+\frac{1}{2}aE^\mathcal{M}_{\vec{0}}+3\left(1-\cos\left(\frac{a\vec{p}}{\sqrt{3}}\right)\right)\right]}.
\end{equation}
With this formulation, we can use equation \ref{eq:twist_angle} to write a better function which assigns priors to all non-zero-momentum energy priors:
\begin{equation}\label{eq:non_trunc_dispersion}
    P[aE^\mathcal{M}_{\vec{p}}] = \cosh^{-1}{\left[1+\frac{1}{2}P[aE^\mathcal{M}_{\vec{0}}]+3\left(1-\cos\left(\frac{a\vec{p}}{\sqrt3}\right)\right)\right]}\times\left[ 1 + P[\epsilon^\mathcal{M}] \left( \frac{a\vec{p}}{\pi} \right)^2     \right].
\end{equation}
Implementing this change to our fitting routine did not remedy the disagreeing meson energy priors we encountered in test fitting.  We discuss these issues in chapters \ref{cha:DBtopi} and \ref{cha:DsBstoK}.  In test fitting we found that meson energy priors assigned via equation \ref{eq:dispersion_energy} and those assigned via equation \ref{eq:non_trunc_dispersion} were consistent with each other, as were the fit posteriors in either case.  Our final decision in this matter was to us the non-truncated dispersion relation (equation \ref{eq:non_trunc_dispersion}) as part of our finalized fitting methodology.   


\subsection{Excited state meson energies}\label{sec:excited_state_energies}
Up until this point, the catalog of meson energies discussed has remained limited to those whose exponential order $i=0$ in equation \ref{eq:2ptcorrfit}.  Having done so, we are now positioned to discuss all meson energy states, oscillating or otherwise, which have an exponential order $i\neq0$, $i\leq N_\mathrm{exp}$, where $N_\mathrm{exp}$ is maximum number of exponentials for which we model in our correlator fitting routine.

 Referring back to equation \textbf{SECTION REF INFINITE TOWER OF STATES}, there are in theory an infinite number of higher order states that might contribute to a correlation function.  Fortunately, these higher order states are constructed such that they are well behaved: their amplitudes are such that their contributions to the correlation function decay much more quickly along the temporal length of the lattice.  The effective mass plots in figure \ref{fig:effective_mass_plots} shows this behavior: the higher order states decay and the data is dominated by the $i=0$ ground state after some time $t/a$.  This well behaved ordering allows us to constrain the number of exponentials we fit to a manageable number. We do this by determining a point along the temporal length of the lattice $t_\mathrm{min}$, such that any correlator data $C^\mathcal{M}_2(t<t_\mathcal{min})$ is ignored in our fitting rouine.  With an appropriate value for $t_\text{min}$, we can estimate that the influence of all higher order states of $i > N_\mathrm{exp}$ decay before $t_\mathrm{min}$.  These states can be either ignored, or be thought of as "folded into" the signal of the $i = N_\mathrm{exp}$ state.  In other words, we can truncate the infinite equation (equation \ref{eq:meson_correlator}) to some small number of exponentials $N_\mathrm{exp}$.   We discuss our method of selecting appropriate values of $N_\mathrm{exp}$ and $t_\mathrm{min}$ later in section \ref{sec:Nmax_and_tmin}. 

 Let us first consider the $i=1$ energy states for the pion and the kaon.  Following the similar rationale of determining the energy difference between the ground state and lowest lying oscillating state of these mesons, we can utilize the expected energy splittings between the non-oscillating $i=0$ and $i=1$ states.   In the physical continuum, the states which are comparable to the $i>0$ states on the lattice are those with a greater energy but the same quantum numbers as their ground state counterparts($J^p = 0^-$ for the ground state psuedoscalar pion and kaon).  From the PDG, the first radial excitation of the ground state pion we might associate with the lattice $E^\pi_{i=1}$ state is the $\pi(1300)$.  Similarly, the candidate particle for $E^K_{i=1}$ on the lattice would be the $K(1460)$.  Following these values we might expect an energy difference between the $i=0$ and $i=1$ states to be $\Delta E^\pi_1 \approx 1160$ MeV and $\Delta E^K_1 \approx 965$ MeV.  The experimental decay widths of these radially excited states are broad relative to their ground state counterparts, and again, are not necessarily the $i=1$ states that appear in our pion and kaon correlation functions.  The effects of chiral symmetry breaking in relation to our physical and nonphysical light quarks on the lattice could also affect an expected mass splitting.  Conservatively, we can say that we might expect the mass splitting between the $i=1$ and $i=0$ states for both the pion and kaon to be of the order of $2\times\Lambda_\mathrm{QCD}$.  We therefor assign the central values for $P[E^{\pi(K)}_{i=1}]$ equal to the central value of $P[E^{\pi(K)}_{i=0}] + (2\times\Lambda_\mathrm{QCD})$.  We then assign an appropriately broad prior uncertainty of $\pm0.5\times\Lambda_\mathrm{QCD}$. For $i=1$ light meson energies with non-zero-momentum, we can once again utilize the dispersion relation discussed in the previous section.  In test fitting we find that these priors do not overly constrain the posterior fit results for the $i=1$ light meson energies, which often have standard deviations of order ten times smaller than their matching priors.

Unlike the light mesons, the heavy mesons $H$, $H_s$ don't have as clear of an expected mass splitting between the $i=0$ and $i=1$ sates.  We have some theoretical continuum expectations at both the charm and bottom ends of the heavy quark spectrum: $\Delta E^D_{1} \approx 710$ MeV, $\Delta E^{D_s}_1 \approx 720$ GeV, $\Delta E^B_{1} \approx 610$ MeV, and $\Delta E^{B_s}_1 \approx 610$ GeV \cite{Ebert:2009ua}.  Of these four, only the $D$ meson has strong enough experimental evidence for a well defined first radial excitation, $D_0(2550)^0$.  This suggests a value for $\Delta E^D_{1} \approx 680$ MeV.  Regardless, these values are given in the continuum for physical heavy meson, and are not necessarily the exact $i=1$ states that appear in our heavy meson correlation functions.  Given this lack of clarity we resort to using the natural energy scale of QCD, $\Lambda_\text{QCD} = 500$ GeV as the expected mass splitting.  Like with the light mesons, we define $P[E^H_{i=1}]$, $P[E^{H_s}_{i=1}]$ relative to their $i=0$ counterparts, such that the central value of $P[E^{H_{(s)}}_{i=1}]$ is equal to the central value of $P[E^{H_{(s)}}_{i=0}] + \Lambda_\mathrm{QCD}$.  We likewise assign a broad prior uncertainty of $\pm0.5 \times\Lambda_\mathrm{QCD}$.  In test fitting we find this prior assignment does not overly constrain our fit results, which often have standard deviations of order ten times smaller than their matching priors.

For all other higher order energy states, that is non-oscillating $i>1$ and oscillating $i>0$, we default to using the natural energy scale of $\Lambda_\mathrm{QCD}$ to assign priors for energy splittings.  This, in conjunction with prior uncertainties of $\pm0.5 \times\Lambda_\mathrm{QCD}$ leads to priors with large uncertainties that cover the whole upper range of possible exited state energies. 

%$\Lambda_\text{QCD}$ as defined for use in prior calculation is a \verb|gvar| object with a broad width equal to half its central value. 

\subsection{Ground state meson amplitudes}\label{sec:2pt_amps_ground_state}
Having discussed meson energies and their priors at length, we can now move on to discuss the other parameters that appear in our two-point correlator fit function: the amplitudes $A_i^{\mathcal{M},n}$, $A_i^{\mathcal{M},o}$. To determine the priors we assign to meson masses (which equals their energy in the case of no momentum) we used a custom function to find where the change in a correlator's effective mass was minimized.  We can similarly for a zero-momentum meson correlator whose effective amplitude $A_\mathrm{eff}(t)$, in the limit of large $t$, can be written as
 \begin{equation}\label{eq:2ptaeff}
    A_{\mathrm{eff}}(t) = \sqrt{\frac{C_2(t)}{e^{-M_{\mathrm{eff}}t}+e^{-M_{\mathrm{eff}}(N_t-t)}}}.
\end{equation}
Here $N_t$ is the temporal length of the lattice in lattice units, and $M_\mathrm{eff}$ is the effective mass at lattice time slice $t$.  We have dropped the implied notation that the masses, amplitudes, and correlators have matching meson superscripts $\mathcal{M}$, and that the effective masses are also functions of time. Figure \ref{fig:2ptAmps} shows a similar picture to what we see in effective energy plots, the signal from the higher order states die out after a certain $t/a$, and a stable ground state plateau is resolvable.  

Like before, we assign central values to the priors $P[A^\mathcal{M}_{i=0}]$ equal to the value of $A^\mathcal{M}_\mathrm{eff}(t)$ where the change in its rolling average is minimized.  Similarly we then assign fractional uncertainties to these priors appropriate to the relative size of the standard deviations of the plateau-region data in the the affective amplitude plots.  Tables \ref{tab:Hpimeson_mass_and_amp_priors} and \ref{tab:HsKmeson_mass_and_amp_priors} give the fractional uncertainties we assign to the ground state, zero-momentum mesons across all ensembles for either decay channel. 

With four different heavy quark options for each heavy meson, we opt to not assign separate priors for each option.  We noticed upon examination of their effective amplitude plots that the relative uncertainty of $A^\mathcal{M}_\mathrm{eff}(t)$ grew as the the mass of heavy quark in the meson grew.  We instead determine a fractional uncertainty from the effective amplitude plot of the lightest heavy meson, and scale that uncertainty up according to the ratio of the given heavy quark mass to the lightest mass option.  This amplitude scaling is defined via equation \ref{eq:Hmeson_amp_scaling}:
\begin{equation}\label{eq:Hmeson_amp_scaling}
\Delta(A_{m_j}) = \Delta(A_{m_0}) \left[\frac{(R-1)(\frac{m_j}{m_0}-1)}{(\frac{m_3}{m_0}-1)} + 1  \right].    
\end{equation} 
Here $m_j$ for index $j\in\{0,1,2,3\}$ equals the lightest to heaviest heavy quark mass on any ensemble, where $m_0$ is the lightest and $m_3$ is the heaviest.  $\Delta(A_{m_j})$ is the fractional uncertainty assigned to the prior for the amplitude of the heavy meson which contains the heavy quark of mass $m_j$.  The desired ensemble specific ratio $R$ is selected to be equal to the desired ratio of $\Delta(A_{m_3})/\Delta(A_{m_0})$.  For example, if $\Delta(A_{m_0})$ is determined to be $0.10$ and the desired value of $\Delta(A_{m_3})$ is to be $4\times\Delta(A_{m_0})=0.40$, then we set $R=4$.  The fractional uncertainties of $\Delta(A_{m_1})$ and $\Delta(A_{m_2})$ meanwhile will land somewhere between $0.10$ and $0.40$, scaled in relation to the ratio of their heavy quark mass $m_{1,2}$ to $m_0$.  In test fitting we find that this method of prior uncertainty assignment produces broad, conservative priors which avoid violating the 10:1 prior to posterior uncertainty ratio benchmark.  

%This is, admittedly, a somewhat convoluted solution to a non-existent problem.    

 %These two-point amplitudes also have oscillating and higher order counterparts, however the determination of these is not easily done simply by evaluating effective amplitude plots such as figure \ref{fig:2pt_amps_fine}.  In this figure, the effective amplitudes of both the mother and daughter mesons, for all mass and twist combinations, are presented for a single ensemble.  From plots such as this, the only prior we can determine from "eye-balling it" is a non-oscillating ground state amplitude for both mesons.  %Additionally, its evident that the \BtoPi\space and \BstoK\space effective amplitude plots on the same ensemble demonstrate similar behaviors: a effective amplitude that stabilizes just before $t/a = 10$ at about $A_\text{eff} = 0.250$.  


% Using this effective amplitude function, we can use a similar algorithm to the one discussed in section \ref{sec:meson_rest_masses} to find where the rolling average of consecutive values of $A_{\text{eff}}(t)$ changes the least.  
% NEED TO DOUBLE CHECK HOW THIS ACTUALLY WORKS

% In testing, we found that in the minority of cases, some algorithmically determined amplitude priors are given uncertainties considerably larger than their central values.  In other words, this automated method may assign an amplitude prior of $0.2 \pm 0.4$, wherein the uncertainty is twice the size of the magnitude of the central value.  Our two-point correlators are generated such that their ground state amplitudes are always positive [CITATION NEEDED], and so we want priors that discourage negative central values for ground state amplitudes.  In such cases, we override the algorithmically derived prior with a more conservative, negative central value discouraging prior that is manually determined from effective amplitude plots.  This manual override prior is broad enough to encompass the reasonable\footnote{Reasonable here denotes the plateau range in effective amplitude plots that occurs after the excited state signal has decayed, and before the signal noise becomes to great nearer the middle of the lattice length.} effective amplitude ranges of all mass-twist combinations for all mother and daughter mesons on a single ensemble. 

\begin{figure}
    \centering
    \includegraphics[width=0.9\linewidth]{chapter4/pdfs/CustomAmps.pdf}
    \caption{Sample pion effective amplitude plots on the f5 and uf5 ensembles.  The effective amplitude $aM\mathrm{eff}(t)$, which is calculated from equation \ref{eq:2ptaeff} and its respective two point correlation function $C^\pi_2$, is shown in blue. The rolling average of the effective mass R.Avg$(t)$ is shown in dashed green. Where the change in this rolling average is minimized $\Delta_\mathrm{min}$ of R.Avg$(t)$ is shown in dotted red. Consecutive red bands show regions of $\pm5\%$, out to $\pm30\%$, of $\Delta_\mathrm{min}$ of R.Avg$(t)$.}
    \label{fig:2ptAmps}
\end{figure}

\subsection{Non-ground state and non-zero twist meson amplitudes}\label{sec:2ptamps_non_ground}
Having discussed assigning priors to the amplitudes of zero-momentum, non-oscillating, exponential order $i=0$ mesons, we can now go on to discuss how we assign priors to the rest of the amplitudes which appear in our two-point correlator fit function (equation \ref{eq:2ptcorrfit}).  Let us begin with examining the amplitudes of non-oscillating $i=1$ mesons with non-zero-momentum $P[A^{\mathcal{M}}_{\vec{p}}]$.  We can likewise employ a relativistic dispersion relation to assign these priors:
\begin{equation}\label{eq:dispersion_amplitude}
P[A^{\mathcal{M}}_{\vec{p}}] = P[A^{\mathcal{M}}_{\vec{0}}] \times\sqrt{\frac{P[E^\mathcal{M}_{\vec{0}}]}{P[E^\mathcal{M}_{\vec{p}}]}} \times \left[ 1 + P[\delta^\mathcal{M}] \left( \frac{a\overrightarrow{p}}{\pi} \right)^2     \right].
\end{equation}
Here we introduce another discretization parameter $\delta^\mathcal{M}$ which performs the same function as $\epsilon^\mathcal{M}$ in equation \ref{eq:dispersion_energy}.  We note that in this construction, $P[A^{\mathcal{M}}_{\vec{p}}]$ becomes correlated with $P[E^\mathcal{M}_{\vec{p}}]$. Across all ensembles and both decay channels we set $P[\delta^\mathcal{\pi}]$, $P[\delta^\mathcal{K}] = 0.0(1.0)$.  In test fitting we find that the fit posteriors for $A^{\mathcal{M}}_{\vec{p}}$ and $\epsilon^\mathcal{M}$ are consistent with their matching priors.

The remaining amplitude terms to account for are those with oscillating and or $i>0$ components, which we can collectively refer to as $A^*$. To determine the priors for these, we must resort to a $\log(\mathrm{GBF})$ testing regime as we have no physics inclination for what they ought to be.  As a starting point, we set the central value of these priors equal to that of their non-oscillating $i=0$ counterpart, with an initial fractional uncertainty of $\Delta(A^*)=1.0$.  We introduce this notation $\Delta(A^*)$ to indicate that we use one parameter to set the fractional uncertainty of all these amplitudes.  We find that in $\log(\mathrm{GBF})$ testing, fractional uncertainties larger that 1.0 are preferred across all ensembles and both decay channels.  The practical meaning of this finding aligns well with our expectations: we have little \textit{a priori} knowledge on what values these amplitudes should be in out fit routine.  Therefore having priors for these amplitudes with large fractional uncertainties permits the correlator data itself to maximally inform the fitter.  The exact values of $\Delta(A^*)=1.0$ used in our final fit are given in tables \ref{tab:Hpimeson_mass_and_amp_priors} and \ref{tab:HsKmeson_mass_and_amp_priors}.

\section{Three-point correlation function priors}\label{sec:3pt_priors}
In the following subsections, I describe how I determine a full set of three-point amplitude priors, which include ground states, oscillating states, and higher order states. For a given three-point current correlator, its fit is proportional to the summed permutations of the amplitude terms $J_{ij}^{kl}$, where $i,j \in \{0,1,...,N_{max} - 1\}$, and $k,l \in \{n,o\}$.  Here $n$ denotes a non-oscillating state (rather than some order of exponential) and $o$ denotes an oscillating state.  Equation \ref{eq:3pt_correlator_fit} gives the full fit equation for $H\rightarrow\pi$.  A three point correlator for $H_s\rightarrow K$ is fit similarly.   
\begin{multline} \label{eq:3pt_correlator_fit}
    C_3^{\pi,H}(t,T) = 
    \sum_{i,j=0}^{N_\text{exp}-1} \Bigl[A^{\pi,n}_i {J_{ij}^{nn}} A^{H,n}_j e^{-E^{\pi,n}_it} e^{-E^{H,n}_j(T-t)} 
     -(-1)^{(T-t)} A^{\pi,n}_i J_{ij}^{no} A^{H,o}_j e^{-E^{\pi,n}_it} e^{-E^{H,o}_j(T-t)} \\
    -(-1)^t A^{\pi,o}_i J_{ij}^{on} A^{H,n}_j e^{-E^{\pi,o}_it} e^{-E^{H,n}_j(T-t)} 
     +(-1)^T A^{\pi,o}_i J_{ij}^{oo} A^{H,o}_j e^{-E^{\pi,o}_it} e^{-E^{H,o}_j(T-t)}\Bigr].
\end{multline}
The $J_{nn}^{00}$ term, or ground state amplitude, in this equation most important fit parameter for this work, as it is this parameter that is used in form factor construction.  In our fitting approach however, the $J_{nn}^{00}$ term is fit along side all other two and three point function related terms simultaneously.  This simultaneous fit captures, in principle, all correlations among these various fit parameters.  It is for this reason that we ensure our priors for two point function amplitudes and energies are appropriately determined: they \textit{can} have an effect on the fit posterior of the $J_{00}^{nn}$ term.

\subsection{Determining non-oscillating ground state three-point amplitude priors}\label{sec:Amp_effs_3pt}
In a similar fashion to the determination of ground state two-point amplitudes, we can use effective three-point amplitude plots to determine widths and central values for their respective priors.  To be more specific, effective three-point amplitude plots allow us to estimate the prior for the $J_{00}^{nn}$ term (where $n$ denotes a non-oscillating state, and zero denotes a state of exponential order $i=0$) in the equation \ref{eq:3pt_correlator_fit}.  For a given three-point correlator, we can construct its effective amplitude from the following equation:
\begin{equation}\label{eq:3ptVeff}
    J_{\text{eff}}(t,T) = C_3(t,T)\frac{\exp\left[{M_{\text{eff}}^{\pi,K}t +M_{\text{eff}}^{B_{(s)}}(T-t)}\right]}{A_{\text{eff}}^{\pi,K} \times A_{\text{eff}}^{B_{(s)}}}.
\end{equation}

For three-point amplitudes, we cannot use the same methods of prior determination that we used for determining two-point amplitudes.  There is no expected dispersion relation we expect three-point amplitudes to follow.  In fact, the $q^2$ dependence of these three-point amplitudes is a behavior that emerges from the data alone, rather than by any strict prior enforcement.  Furthermore, as seen in equation \ref{eq:3pt_correlator_fit}, the value of a three-point correlation function for a given $t,T$ has contributions from combinations of various oscillating and non-oscillating amplitudes, even in the case where $i,j = 0$.  Additionally, These oscillating amplitude components can have either positive or negative sign.  The implication then is that even in cases where an effective three-point amplitude appears to achieve a steady plateau in some $t$ range, it cannot be assumed that the $J_{00}^{nn}$ amplitude is the only contributing state to that steady plateau.  While in correlator fit testing it was often the case that the fit posterior for a given $J_{00}^{nn}$ was consistent with some $J_{\text{eff}}(t,T)$ plateau, it was not universally true.  For these reasons we cannot justify as narrow prior widths for well resolved ground state thee-point as for similarly well resolved two-point amplitudes.


With no dispersion relation on which to base non-zero-twist three-point amplitude priors, and a desire to not unduly influence the $q^2$ dependence their corresponding fit posteriors, for much of this project we adopted a near-maximally conservative prior determination methodology.  For a given current component $J \in \{S,V,X,T\}$, on a given ensemble, we opted to use a single prior central value and uncertainty for all heavy-quark mass and twist combinations.  This central value and uncertainty was determined by visually looking at a plot (like figure \textbf{INSERT REF TO PLOT}) which included the effective amplitudes of all mass and twist combinations for a given current component. Such a plot might include twenty effective three-point amplitude plots, from which a visual grouping of plateau behavior might be resolved with varying degrees of clarity.  A central value would be determined from some apparent mean of these plateauing effective amplitudes, and the uncertainty would be determined such that both the upper and lower bounds of the plateau grouping were encompassed by the resultant prior bounds.  

This prior determination methodology was utilized through most this research project and in many acceptable correlator fits.  This methodology however is somewhat flawed when we consider a number of factors.  First, this method of prior determination does not include \textit{any} consideration of $q^2$ dependency for three-point amplitudes.  While this choice is preferable to over-constraining any $q^2$ dependency, better fit results might be achieved by having \textit{some} $q^2$ dependency built into our priors.  Second, this method somewhat misconstrues the function of a prior's uncertainty as encoded in a \verb|gvar|.  By choosing to encode a prior's central value and uncertainty in a \verb|gvar|, we encode that this central value is the mean of some normal distribution whose standard deviation is equal to the prior's uncertainty.  This is no revelation, but it means that the mindset, where any value within a prior's uncertainty range is equally preferable and any value outside of it is discouraged, is not coherent with the gaussian structure of the prior as encoded by a \verb|gvar|.  

This issue becomes more problematic when we consider how, as shown by figure \textbf{INSERT PLOT REFERENCE}, the average uncertainty or noise of high twist (low $q^2$) effective amplitudes is greater than that of low twist (high $q^2$) effective amplitudes.  Figure \textbf{PLOT REFERENCE} then also shows the general trend that effective amplitudes plateaus tend to decrease in value as twist increases. These effects combined mean that, as displayed in figure \textbf{PLOT REFERENCE}, for a given current component and heavy-quark mass, effective amplitude plateaus are larger and more stable a low twists, but are smaller and less stable at high twists.  Applying a gaussian prior, whose central value is a rough mean of this whole range of amplitudes, can create an undesirable fit outcome. In effect, the gaussian prior structure favors fit posteriors closer to the given mean or central value.  The stable plateau of a zero or low twist effective amplitude might be so well resolved that this pull towards the prior's central value has little effect in the fitter's calculation of that amplitude's fit posterior.  Conversely, at the other end of the prior uncertainty bounds, the noisy and often unstable plateaus of the corresponding high twist amplitudes will be more influenced by this pull towards the mean.  In other words, the more indeterminable a fit parameter is from the data alone, the more influence the matching parameter's prior has on the final fit posterior.  This phenomena is intended by the constructed of our augmented $\chi^2$ procedure.  Taking this all into consideration, we really ought to have different priors for the three-point amplitudes of different twist but matching current components.  We stress matching current component here because, for a given current component, the effective amplitudes plots vary in plateau position and uncertainty much more as twist is varied rather than as heavy-quark mass is varied.  

With the desire to have twist-specific priors for three-point amplitudes, we still do not want to venture into the territory of unfairly constraining the expected $q^2$ dependency.  Therefore while we determine central values by visually identifying where an effective three-point amplitude plateaus, we simply assign an uncertainty at least equal to the associated central value.  In other words, all ground state three-point amplitude priors have at least $100\%$ uncertainty.  This method is consistent with not overly constraining the expected $q^2$ dependent behavior, and avoids the "pull towards the mean" problem of unstable high twist amplitudes given the same prior as stable low twist amplitudes.  We stress "at least" here because, in some cases, an effective amplitude plot was so noisy or unstable that we felt $100\%$ uncertainty could still be too constraining.  Such cases are clearly indicated in tables \ref{tab:3pt_f5_priors}-\ref{tab:3pt_uf5_priors}.

\begin{figure}
    \centering
    \includegraphics[width=0.75\linewidth]{chapter4/pdfs/0tw.pdf}
    \includegraphics[width=0.75\linewidth]{chapter4/pdfs/2tw.pdf}
    \caption{Caption}
    \label{fig:3pt_effective_amps}
\end{figure}

% Figure \ref{fig:3pt_amps_fine} gives a sample \BtoPi\space scalar current effective amplitudes plot, which covers all mass and twist combinations. From plots such as this it is straightforward to visually (manually) estimate $J_{00}^{nn}$ priors.  For this research, we choose to use one prior to apply to all mass, twist, and source-sink separation $T$ combinations for a single ground state three point amplitude. This choice necessitates priors with relatively large uncertainties in order to comfortably cover the effective amplitude space spanned by all variable combinations.  In particular, a $J_{00}^{nn}$ prior's broad uncertainty comes mostly from our choice of imposing that it applies all twist options.   

% In theory, we \textit{could} use different priors to apply to different twist options (different momentum transfers, different $q^2$ in other words). From effective amplitude plots like figure \ref{fig:3pt_amps_fine} it might even appear reasonable to do so; it is possible to distinguish between five different effective amplitude plateaus for the five twist options.  However, its important to remind ourselves that the ultimate aim of this work is to calculate the form factors of \BstoPiK\space as a function of $q^2$.  A three point amplitude's twist dependence \textit{is} the basis of a form factors $q^2$ functionality\footnote{In another journal I am likely to detail the steps of this basis in more detail.}.  For this reason, we wish not to overly constrain the fitting of ground state three point amplitudes in relation to their momentum dependence.  A three point amplitude's momentum transfer dependence should be self evident from the fit results, rather than enforced \textit{a priori}.  With that said, we still ensure that these conservative prior uncertainties avoid violating the 10:1 prior to posterior rule of thumb before.  Table \ref{tab:3pt_priors} gives the current working values of these manually determined $J_{nn}^{00}$ priors.

\subsection{Determining oscillating and excited state three-point amplitude priors}
Unfortunately, determining the magnitudes of the oscillating and non-ground state three-point amplitudes cannot be simply achieved by observing effective amplitude plots.  Furthermore, for every one $J_{00}^{nn}$ state, there are \textit{many} more oscillating and higher order exponential states $J_{ij}^{kl}$ where $\{i,j\}\neq \{0,0\}$ and or $k,l \neq n$.  In fact, if $N_\text{exp} = 4$, for every one $J_{00}^{nn}$ state in a fit, there are 63 other states which contain oscillating components and or higher order exponentials.  This gives us 64 total fit parameters for every combination of current insertion, heavy quark mass, and daughter meson momentum each. Determining what the priors for these 63 non-ground states ought to be then poses an obvious challenge.  There is no algorithm or effective amplitude plot-based determination we can use;  we have no \textit{a priori} physics knowledge, other than knowing they can be positive or negative.  Thankfully, that sign indifference means we can assign a value of zero to the central value of all these non-ground states' priors.  

This still leaves us with the challenge of determining and assigning prior uncertainties for thousands of fit parameters.  Following from the similar work of \cite{Will_technical}, we can first try grouping non-ground state priors into two categories.  Per ensemble, for a given current component $J$, we can assign a single prior to all states which contain any higher order exponentials $J_{ij\neq00}^{kl}$.  This method leaves us only with the lowest lying oscillating states $J_{00}^{kl\neq nn}$, to which we can then assign a separate prior.  We can refer to these two groupings as $J_{\neq0}$ and $J_\mathrm{osc}$ as shorthand, the priors for which we can label $P[J_{\neq0}]$ and $P[J_\mathrm{osc}]$ respectively.   

With no prior physics knowledge to set the uncertainties of $P[J_{\neq0}]$ and $P[J_\mathrm{osc}]$, we used the highest uncertainty of a matching $P[J_{00}^{nn}]$ as an initial guess.  We then utilized the same $\log (\mathrm{GBF})$ optimization method discussed in section \ref{sec:prior_determination_methodology} to refine these guesses into final working values, which are listed in tables \ref{tab:3pt_pesky_priors_Hpi, tab:3pt_pesky_priors_HsK}.


% These "pesky" priors are denoted as \verb|V0| and \verb|Vn| in table \ref{tab:3pt_priors} and in the code packages I use in this work.  These terms respectively match $J_{00}^{kl}$ for $k$ and or $l \neq n$, and $J_{ij}^{kl}$ for any $i,j \neq 0$.  Alternatively in words, $V0$ denotes all lowest order state to lowest order state three-point amplitudes that contain oscillating components, and $Vn$ denotes all three-point amplitudes containing higher order components, oscillating or non-oscillating.

 % Therefor we use the a similar GBF optimization strategy as described before.  We do however still need an initial set of \verb|V0| and \verb|Vn| priors to begin GBF testing. Considering that these amplitudes can all in theory take negative values, it is natural to set the central values of these pesky priors as $0.0$.  As for their uncertainties, we assign their initial values as equal to their respective ground state non-oscillating (non-pesky) counterpart.  From this baseline, we proceed with GBF optimization testing, aborting the procedure when we come close to violating the 10:1 prior to posterior uncertainty rule.  In testing we find that the uncertainty of the GBF optimized \verb|V0| prior tends to (but not always) be at least two times larger than the uncertainty on its matching \verb|Vn| prior.  The fit posteriors to which these priors correspond to are not carried forward in the following form factor calculations, so it might be tempting to not give too much care to the priors.  However, in the domain of fitting correlation functions, our fitter does not unfairly prioritize any $J_{ij}^{kl}$ over another.  The non-oscillating ground state $J_{00}^{kl}$ is only one $J_{ij}^{kl}$ of many in our fit function (equation \ref{eq:3pt_correlator_fit}).

In test fitting, even after GBF optimizing the priors for oscillating and higher order three point amplitudes, we found two consistent outlier cases where fit posteriors still differed from their respective priors by two or more standard deviations.  These outlier cases occurred irrespective of of mass, twist, or current component.  The first of these cases were the $J_{00}^{on}$ states.  The second of these cases were the states $J_{10}^{kl}$ for all $k,l \in \{n,o\}$.  This suggested that our initial non-ground state prior grouping method could be improved.  We therefore introduced an ensemble specific, but otherwise global, multiplicative factor $\sigma_\mathrm{ens}$ which effectively multiplies or \textit{widens} the uncertainty on these special case.  We then underwent the same GBF optimization for this special prior widener for each ensemble $\sigma_\mathrm{ens}$, yielding values that span roughly from two to five across the five ensembles, and can also be found in tables \ref{tab:3pt_pesky_priors_Hpi, tab:3pt_pesky_priors_HsK}. This entire process is somewhat "in the weeds" for standard correlator fitting prior selection.  And while implementing these $\sigma_\mathrm{ens}$ terms does not appear to have any measurable impact on the ground state amplitudes we carry forward into our form factor calculations, it does nevertheless accomplish two things.  First, it does make our correlator fitting routine slightly more conservative, that is, the affected priors are wider and less constraining to the fit.  Second, the implemented change does meet the $\Delta \log(\mathrm{GBF}) > 3$ threshold wherein we can state the change is significant. 

% Finally, I note that the correlations among scalar current components of matching masses or matching twists tend to be the greatest subset of correlations withing one matrix.

% \subsubsection{Ensemble specific findings}
% Looking at these sample mass correlation matrices more broadly, the ensembles with the largest off-diagonal correlations are, in order of largest to smallest: Fine, Fine-physical, Superfine, Ultrafine, Superfine-physical.  Among this set of sample correlation matrices, the highest degree of correlations occurs in the Fine-physical \BtoPi\space sample matrix (\ref{fig:CorrMtrx_Fp}) between the the most massive and second most massive zero-twist scalar current components, a correlation equal to $0.46$.  Additionally, all the zero-twist scalar current components in this matrix correlate with each other at values ranging from $0.39$ up to $0.46$.  There are similarly high correlations among the zero-twist scalar current components for the Fine ensemble \BtoPi\space matrix (figure \ref{fig:CorrMtrx_F}): $0.3$ to $0.4$.  

% Alternatively, the maximum-twist scalar current components have the highest correlations between each other in the \BstoK\space matrices on the Fine (figure \ref{fig:CorrMtrx_Fs}) and Fine-physical ensembles (figure \ref{fig:CorrMtrx_Fps}): ranging from $0.37$ to $0.42$, and $0.27$ to $0.33$ respectively.

\section{Testing auxiliary fit parameters}\label{sec:other_fit_params}
\subsection{Testing $N_\mathrm{exp}$, $t_\mathrm{min}$, and $t_\mathrm{max}$}\label{sec:Nmax_and_tmin}
In addition to those discussed already, there are other parameters we can adjust to improve our fit results.  The first we can discuss here are the total number of exponentials $N_\mathrm{exp}$ which we include in our fit forms (equations \ref{eq:2ptcorrfit} and \ref{eq:3ptcorrfit}).  These higher order exponential states have higher energies, and as a result decay quickly in our correlation functions.  Therefore, after some minimum time $t_\mathrm{min}$, only a few exponential states are likely contributing significantly to the correlation function.  For this reason we tie together the study of $N_\mathrm{exp}$ and $t_\mathrm{min}$ in this research.

Adjusting these two auxiliary fit parameters presents trade-offs worth investigating.  In considering the number of exponentials, too small of a value for $N_\mathrm{exp}$ might yield poor fit results, as there are not enough parameters (energies $E_i$ and amplitudes $A_i$ for example) to properly model the data.  Meanwhile, increasing $N_\mathrm{exp}$ adds more parameters to the fit, whose contributions to the data might likely have decayed before $t_\mathrm{min}$, or are otherwise indeterminable by the fitter.  This can increase the fit computation time without improving the results. Ideally we wish to select a value of $N_\mathrm{exp}$ such that test fit results are stable, in that increasing or decreasing $N_\mathrm{exp}$ by one does not significantly affect ground state posteriors.

Meanwhile, adjusting $t_\mathrm{min}$ can have similar trade-off effects.  Any correlator data of $t<t_\mathrm{min}$ is not included in the correlator fit.  By increasing $t_\mathrm{min}$, we are effectively throwing out more and more data correlator data, which often can lead to larger uncertainties on fit posteriors.  So while reducing $t_\mathrm{min}$ can in theory lead to smaller uncertainties, it also increases the likelihood of unwanted contributions to the data from higher order exponential states.  Like testing values of $N_\mathrm{exp}$, we wish to select values of $t_\mathrm{min}$ where adjusting by $\pm1$ does not significantly affect fit posteriors.  Additionally, it is worth noting that we can set different values of $t_\mathrm{min}$ for different correlation functions, both two and three-point.  This is also true across different ensembles and both meson decays.  We tend to find however that a single value of $t_\mathrm{min}$ for all two-point correlators and another for all three-point correlators is sufficient for a given ensemble, for either meson decay.  

Figures x and x

Where $t_\mathrm{min}$ 


\subsection{Testing SVD cuts}\label{sec:SVD}
SVD cut basis from \cite{Dowdall_2019}

 