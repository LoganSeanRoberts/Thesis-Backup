\chapter{Calculating Form Factors}\label{cha:form_factors}
In this chapter I will provide the mathematical bases for turning the results of our correlator fits into continuous form factors results.  The chapter outline is as follows:
\begin{itemize}
    \item \textbf{Constructing Scalar, Vector, And Tensor Form Factors}. Here I will likely present the key form factor equations \ref{eq:scalar_FF}, \ref{eq:vector_FF}, and \ref{eq:tensor_FF}.  I will define the appropriate variables, and note that certain terms will be discussed in the following sections. 
\end{itemize}
\begin{equation}\label{Eq:fitnormalisation}
  \bra{L}J_{\mathrm{latt}}\ket{\brachat{H}}=2Z_{\mathrm{disc}}\sqrt{M_{H}E_{L}}J^{\mathrm{nn}}_{00}.
\end{equation}
\begin{align}\label{eq:scalar_FF}
  \bra{L}S_{\mathrm{latt}}\ket{H}&=\frac{M_{H}^2-M_{L}^2}{m_h-m_l}f_0(q^2),\\
\label{eq:vector_FF}
  Z_V\bra{L}V^{\mu}_{\mathrm{latt}}\ket{\hat{H}}&=f_+(q^2)\Big(p_{H}^{\mu}+p_{L}^{\mu}-\frac{M_{H}^2-M_{L}^2}{q^2}q^{\mu}\Big)+f_0(q^2)\frac{M_{H}^2-M_{K}^2}{q^2}q^{\mu},\\
\label{eq:tensor_FF}
  Z_T(\mu)\bra{L}T^{k0}_{\mathrm{latt}}\ket{\hat{H}}&=\frac{2iM_{H}p_L^k}{M_H+M_L}f_T(q^2,\mu).
\end{align}
\begin{itemize}
    \item \textbf{Renormalization terms}.  Here I will go into detail on the renormalization terms that are present in the above equations.  These terms include $Z_\text{disc}$, $Z_V$, and $Z_T$.
\end{itemize}
\begin{equation}\label{eq:Zdisc}
    Z_\text{disc} = \sqrt{\text{cosh}(m_{tree})\left(1 - \frac{1+\epsilon_{tree}}{2}\text{sinh}^2(m_{tree})\right)}
\end{equation}
\begin{equation}\label{Eq:Zv}
    Z_V=\frac{(m_h-m_l)\bra{L}S\ket{H}}{(M_{H}-M_{L})\bra{L}V^{0}\ket{\hat{H}}}\Bigg\rvert_{q^2=q^2_{\mathrm{max}}}.
\end{equation}
\begin{itemize}
    \item \textbf{The modified $z$ expansion}.  Here I intend to introduce the motivations for using the modified $z$ expansion in our form factor work.  To deal with the pole mass term at $q^2_{\text{max}} = (M_H - M_L)^2$ we map $q^2$ to the complex $z$ plane in equation \ref{eq:zofq}.  Using the Bourreley-Caprini-Lellouch parameterization we can write our form factor equations as expanded polynomials of $z$, as in equations \ref{eq:scaler_FFofz}, \ref{eq:vector_FFofz}, and \ref{eq:tensor_FFofz}  It would be worth mentioning the impact of choice of $t_0 = 0$, and the enforcement of $f_+(q^2=0) = f_0(q^2=0)$.  A discussion of chiral perturbation theory and the form it takes in the chiral logarithm term is warranted.
\end{itemize}
\begin{equation}\label{eq:zofq}
  z(q^2,t_0)=\frac{\sqrt{t_+-q^2}-\sqrt{t_+-t_0}}{\sqrt{t_+-q^2}+\sqrt{t_+-t_0}}.
\end{equation}
  \begin{align}\label{eq:scaler_FFofz}
    f_0(q^2)&=\frac{\mathcal{L}}{1-\frac{q^2}{M^2_{H_{0}^{*}}}}\sum_{n=0}^{N-1}a_n^0z^n\\
    \label{eq:vector_FFofz}f_+(q^2)&=\frac{\mathcal{L}}{1-\frac{q^2}{M^2_{H^{*}}}}\sum_{n=0}^{N-1}a_n^+\Big(z^n-\frac{n}{N}(-1)^{n-N}z^N\Big)\\
     \label{eq:tensor_FFofz}f_T(q^2)&=\frac{\mathcal{L}}{1-\frac{q^2}{M^2_{H^{*}}}}\sum_{n=0}^{N-1}a_n^T\Big(z^n-\frac{n}{N}(-1)^{n-N}z^N\Big).
\end{align}
\begin{itemize}
    \item \textbf{The Heavy HISQ approach}.  Now in this section I intend to discuss the needs for and the means by which we extrapolate to the physical b-quark mass.
    \item \textbf{Varying the heavy quark mass and HQET}. Here, among other topics, I can describe how the $D_{(s)}\rightarrow\pi (K)$ form factor calculations are a byproduct of Heavy-HISQ.
\end{itemize}


\section{Lattice Form Factors}\label{sec:constructing_form_factors}
With a satisfactory set of correlator fit results, we can work towards calculating lattice form factor results.  We emphasize the term \textit{lattice} here because the form factors directly derived from our correlator fits will still contain ensemble-specific lattice artifacts, namely non-zero lattice spacing and nonphysical quark masses.  Furthermore, these lattice form factors will be a discrete, non-continuous set of form factor values, rather than some function describing a continuous curve.  In section \ref{sec:extrapolate_to_phys_cont} we detail how we calculate physical continuum form factors from these discrete form factors.  For now however, let's examine the outputs of our correlator fitting.

The results of our correlator fitting includes a set two-point function heavy meson masses $M_H$, light meson energies $E_L$,  and both heavy and light meson amplitudes $A_{H,L}$. The results then also include a set of three-point function current insertions with amplitude $J$.  These masses, energies, and amplitudes correspond to different combinations of lattice spacings, quark masses, daughter meson momenta.  Though these quantities have higher order ($n \neq 0$) and oscillating states, it is only the ground states ($n = 0$ and non oscillating) that we need to carry forward into our lattice form factor calculations.

\subsection{Equations of lattice form factors}\label{sec:equations_of _lattice_form_factors}


\subsection{Renormalization Terms}\label{sec:renorm_terms}

\section{Extrapolating to physical continuum form factors}\label{sec:extrapolate_to_phys_cont}



