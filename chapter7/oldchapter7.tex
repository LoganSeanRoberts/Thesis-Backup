\chapter{$B \rightarrow \pi$ and $B_s \rightarrow K$ form factors: an improved determination of $V_{ub}$}\label{cha:BstopiK}

This chapter will likely follow the same template as the previous chapter.  The difference being, with $B\rightarrow\pi$ \textit{and} $B_s\rightarrow K$,  we utilize the Heavy-HISQ approach allows us to use a varied heavy quark mass to extrapolate to physical b-quark masses. A working section structure for this structure is as follows:
\begin{itemize}
    \item \textbf{Introduction}.  Here I can give the context of CKM matrix elements and $V_{ub}$ specifically.  I plan to include a feynman diagram, a discussion of unitarity: $|V_{ub}|^2 + |V_{cb}|^2 + |V_{tb}|^2 = x$, and notes on leptonic and semileptonic determinations of $V_{ub}$.  
    \item \textbf{Correlator Fitting}.  This subject will likely be a truncated discussion of the topics found in chapter \ref{cha:fitting_correlation_functions}, put in the specific context of $B_{(s)} \rightarrow \pi(K)$.  This includes notes on simulation details, correlation functions, ensemble tables, prior choices, an fit stability plots.
    \item \textbf{Modified $z$ Expansion}.  Here I will present the $z$ expansion related functions again but in the context of $B_{(s)} \rightarrow \pi(K)$.  
    \item \textbf{Form Factor Results}.  Taking the work of the previous section I can then produce form factor plots for these meson decays.  
    \item \textbf{Determining $V_{ub}$} Finally we can then take the Form Factor results, combine them with experimental decay rates, and arrive at a calculation of $V_{ub}$. 
    \item \textbf{Discussion and Conclusion}. Finally a short section about summarizes the findings of the previous sections.
\end{itemize}