\chapter{The Standard Model of particle physics}\label{cha:Standard_Model}
\begin{itemize}
    \item \textbf{The Standard Model basis}.  Further discussion would be had on gauge field theories: Abelian and non-Abelian.  QED can be discussed as example of an Abelian gauge theory, while QCD is discussed as a non-Abelian gauge theory.  The meson decays of interest to my work are naturally a product of the electroweak interaction, and so there ought to be a quite detailed section on that.  A brief discussion of the Higgs mechanism would naturally follow. 
\end{itemize}

\section{Relativistic Quantum Fields}\label{sec:RQF}    
    \subsection{The Schrödinger and Klien Gordon Equations}\label{sec:Shrodinger_and_KG_equations}
    We can begin our objective of establishing the Standard Model with the non-relativistic Schrödinger equation, where we can relate the total energy of a particle $E$, its mass $m$, its 3-momentum $\vec{p}$, and its potential energy as a function of space $V(\vec{x})$:
    \begin{equation}\label{eq:non-rel_Shrodinger}
        E = \frac{\vec{p}}{2m} + V(\vec{x}).
    \end{equation}
    To bridge the gap from quantum particle physics to relativistic quantum fields, we can use the relativistic energy-momentum relation: 
    \begin{equation}\label{eq:rel-energy-momentum_relation}
        E^2 = \vec{p}^2c^2 + m^2c^4.
    \end{equation}
    If we now shift our thinking from the quantities of energy and momentum to their operator equivalents acting on a field $\phi(\vec{x}, t)$, we can write
    \begin{align}\label{eq:Hamiltonian}
        E &\rightarrow \hat{H}\phi(\vec{x},t) = i\hbar\frac{\partial}{\partial t}\phi(\vec{x},t),\\
        \label{eq:Laplacian}p &\rightarrow \hat{p}\phi(\vec{x},t) = -i\hbar \nabla \phi(\vec{x},t),
    \end{align}
    where $\hat{H}$ is the Hamiltonian operator, $\hat{p}$ is the momentum operator, $\nabla$ is the Laplacian operator which represents a derivative in all spatial directions, and $\hbar$ is the reduced Planck's constant. Combining equations \ref{eq:rel-energy-momentum_relation} \ref{eq:Hamiltonian}, and \ref{eq:Laplacian} we can write arrive at the Klein-Gordon equation:
    \begin{equation}\label{eq:KGE}
        \left(\frac{1}{c^2} \frac{\partial^2}{\partial t^2} - \nabla^2 + \frac{m^2c^2}{\hbar^2}\right)\phi(\vec{x}, t) = 0.
    \end{equation}
    In natural units ($c = \hbar = 1$), we can write equation \ref{eq:KGE} as
    \begin{equation}\label{eq:KGE-short}
        (\Box + m^2)\phi(\vec{x}, t) = 0,\quad \Box \equiv \partial^\mu\partial_\mu = \frac{\partial^2}{\partial t^2} - \nabla^2.
    \end{equation}
    Arriving at this form of the Klein-Gordon equation presents an issue: this form as derived from equation \ref{eq:rel-energy-momentum_relation} yields negative energy solutions.  This can be seen most plainly by simply by taking the square root of both sides of equation \ref{eq:rel-energy-momentum_relation}: $E = \pm \sqrt{c^2\vec{p}^2 + m^2c^4}$.  This issue persists when we evaluate the probability density ... \textbf{WIP}
    
    This result remains problematic so long as we interpret $\phi(\vec{x}, t)$ as a field describing a single particle, rather than many.  This approach leads us nicely to a discussion of the Dirac equation and Dirac spinors.
  

    \subsection{The Dirac equation and Dirac spinors}\label{sec:Dirac}
    Dirac sought to find an equation whose solution what also a solution to the Klein-Gordon equation, and was linear in $\partial_t \equiv \frac{\partial}{\partial t}$. This would ensure linearity in $\nabla$ should the equation be covariant under a Lorentz transformation.

    Let us begin with the general form of equation \ref{eq:Dirac_general_form}, where we have changed notation $\phi(\vec{x}, t) \rightarrow \psi(t,\vec{x})$ to denote our quantum field's possibility to describe multiple particles, and to match the standard Minkowski spacetime ordering of $(t, x_1, x_2, x_3) \equiv (t, \vec{x})$:
    \begin{equation}\label{eq:Dirac_general_form}
        i\partial_t \psi(t, \vec{x}) = (-i\alpha\nabla + \beta m)\psi(t, \vec{x}),
    \end{equation}
    where $\alpha, \beta$ are some unknown terms which satisfy the linearity constraints mentioned above.  Adopting the Einstein summation convention, if we square equation \ref{eq:Dirac_general_form} we get the following:
    \begin{equation}\label{eq:Dirac_step1}
    \begin{split}
        -\partial_t^2 \psi(t, \vec{x}) &= (-i\alpha\nabla + \beta m)\psi(t, \vec{x})\\
        &=\left( -\alpha^i\alpha^j\partial_i\partial_j - -i(\beta\alpha^i+\alpha^i\beta)m\partial_i + \beta^2m^2 \right)\psi(t, \vec{x}).
    \end{split}
    \end{equation}
    As $\partial_i\partial_j$ is a symmetric tensor, we can write
    \begin{equation}\label{eq:Dirac_step2}
        \begin{split}
            \alpha^i\alpha^j\partial_i\partial_j &= \frac{1}{2} \left(\{\alpha^i, \alpha^j\} +[\alpha^i, \alpha^j]\right)\partial_i\partial_j\\
            &=\frac{1}{2} \{\alpha^i, \alpha^j\}\partial_i\partial_j
        \end{split}
    \end{equation}
    where square and curly brackets denote the commutator and anti-commutator relations respectively.   The Klein-Gordon equation requires the right hand side of equation \ref{eq:Dirac_step1} to equal $\left(-\nabla^2 + m^2 \right)\psi(t, \vec{x})$, and therefor $\alpha$ and $\beta$ must satisfy the following conditions:
    \begin{enumerate}
        \item $\{\alpha^i, \alpha^j\} = 2\delta^{ij}$,
        \item $\beta\alpha^i+\alpha^i\beta = \{\alpha^i, \beta\} = 0$ ,
        \item $\beta^2 = 1$.
    \end{enumerate}

    If $\alpha$ and $\beta$ are simple numbers, we cannot meet the above conditions and therefor solve the general form of equation \ref{eq:Dirac_general_form} or the Klein-Gordon equation.  We will have better luck if we take $\psi(t, \vec{x})$ to be a column vector and take $\alpha, \beta$ to be $n\times n$ matrices.  Under this assumption, the above conditions also impose the restraint that for both $n\times n$ matrices $\alpha^i$ and $\beta$, their traces $Tr(\alpha_i,\beta) = 0$, their eigenvalues $\lambda_{a^i, \beta} = \pm1$, and $n$ must be even. A set of simple $2\times 2$ matrices are unfortunately not sufficient under these conditions. Instead, as a starting point, let us consider a set of three complex, hermitian, unitary, traceless matrices $\sigma$, where
    \begin{equation}\label{eq:pauli_matrices}
        \sigma_i = \begin{pmatrix}0 & 1 \\ 1 & 0\end{pmatrix}, \quad \sigma_j = \begin{pmatrix}0 & -i \\ i & 0\end{pmatrix}, \quad \sigma_k = \begin{pmatrix}1 & 0 \\ 0 & -i\end{pmatrix}.
    \end{equation}
    These matrices are the Pauli matrices, which along with the $2\times 2$ identity matrix $I_2$ allow us to write definitions for $\alpha$ and $\beta$ that satisfy all the above conditions and restraints:
    \begin{equation}\label{eq:def_alpha-beta}
        \alpha = \begin{pmatrix} 0 & \sigma \\ \sigma & 0\end{pmatrix}, \quad \beta = \begin{pmatrix} I_2 & 0 \\ 0 & -I_2\end{pmatrix}.
    \end{equation}
    
    Defining $\alpha$ and $\beta$ in this way gets us to a root of the Klein-Gordon equation, but these are not obviously make equation \ref{eq:Dirac_general_form} Lorentz invariant.  To do so, we can introduce the definition of the Gamma matrices, or Dirac matrices, where $\gamma^0 = \beta$, and $\vec{\gamma} = \beta\vec{\alpha}$.  In this notation we can represent the collection of $\alpha_{i,j,k}$ as $\vec{\alpha} = (\alpha_i,\alpha_j, \alpha_j) \equiv (\alpha_1,\alpha_2, \alpha_3)$.  We therefore have the general gamma matrix $\gamma^\mu = (\gamma^0, \vec{\gamma})$.For completeness we can show the full forms of the gamma matrices like in equation \ref{eq:gamma_matrices}
    \begin{equation}\label{eq:gamma_matrices}
    \begin{split}
        \gamma^0 = \begin{pmatrix}1 & 0 & 0 & 0 \\ 0 & 1 & 0 & 0 \\ 0 & 0 & -1 & 0 \\ 0 & 0 & 0 & -1\end{pmatrix}&, 
        \quad \gamma^1 = \begin{pmatrix}0 & 0 & 0 & 1 \\ 0 & 0 & 1 & 0 \\ 0 & -1 & 0 & 0 \\ -1 & 0 & 0 & 0\end{pmatrix}, \\
        \quad \gamma^2 = \begin{pmatrix}0 & 0 & 0 & i \\ 0 & 0 & i & 0 \\ 0 & -i & 0 & 0 \\ -i & 0 & 0 & 0\end{pmatrix}&, 
        \quad \gamma^3 = \begin{pmatrix}0 & 0 & 1 & 0 \\ 0 & 0 & 0 & -1 \\ -1 & 0 & 0 & 0 \\ 0 & 1 & 0 & 0\end{pmatrix}.
    \end{split}
    \end{equation}
    Here $\gamma^0$ is \textit{time-like} in the Lorentz sense, and is Hermitian: $(\gamma^0)^\dagger = \gamma^0$. Meanwhile for $i=1,2,3$, $\gamma^i$ is \textit{space-like} and anti-Hermitian: $(\gamma^i)^\dagger = -\gamma^i$.  With these matrices we can rewrite equation \ref{eq:Dirac_step2} instead as
    \begin{equation}\label{eq:Dirac_step3}
        \{\gamma^\mu, \gamma^\nu\} = \gamma^\mu\gamma^\nu - \gamma^\nu\gamma^\mu = 2g^{\mu\nu}I_4,
    \end{equation}
    where $g^{\mu\nu}$ is the Minkowski spacetime metric, and $I_4$ is the identity matrix of size $4\times4$.  Finally by multiplying $\gamma^0$ through equation \ref{eq:Dirac_general_form} we can arrive at the Dirac equation, which in covariant form is (in natural units):
    \begin{equation}\label{eq:Dirac}
        \left(i\gamma^\mu\partial_\mu - mI_4 \right) \psi(t,\vec{x}) = 0.
    \end{equation}
    We can simplify the Dirac equation further by adopting the Feynman slash notation where for some $a_\mu$ we define $\slashed{a} = \gamma^\mu a_\mu$.  If we than also generalize across all spacetime dimensions we can write the Dirac equation simple as
    \begin{equation}\label{eq:Dirac_simple}
        (i\slashed{\partial}-m)\psi = 0
    \end{equation}

    At this point we can acknowledge $\psi$ (or really $\psi(t,\vec{x})$) as a column vector, which has more than one degree of freedom.  We can interpret this field which describes many particles $\psi$ as a wave function, and in this instance the wave function for spin-$\frac{1}{2}$ fermions, or \textit{Dirac Spinors}.  At this point we are still presented with the problem of interpreting negative energy solutions which we can see in the Dirac Hamiltonian: $H_\mathrm{Dirac} \equiv i\frac{\partial}{\partial t} = -i\alpha\nabla + \beta m$, where  $\alpha$ and $\beta$ are traceless, and have eigenvalues that sum to zero.  

    To overcome this, we can define the adjoint representation of such a Dirac field as $\bar{\psi} = \psi^\dagger\gamma^0$.  With this construction we can then show that for \textbf{WIP prob density?} 
    \begin{align}\label{eq:Dirac_prob_density1}
        \partial_\mu(\bar{\psi}\gamma^\mu\psi) = \partial_\mu j^\mu = 0 \\
        \label{eq:Dirac_prob_density2}
        \rho = j^0 = \bar{\psi}\gamma^0\psi = \psi^\dagger\psi \geq 0
    \end{align}

    \subsection{Lorentz Transformations}\label{sec:Lorentz_Transformations}
    We can continue developing our relativistic theory by exploring how these fields transform under Lorentz transformations, which we can write as some square matrix $\Lambda$.  Let us begin with a generic 4-vector $x^\mu = x(t, \vec{x})$, with a length $x^2 = t^2 - \vec{x}^2$.  This 4-vector is then invariant under the Lorentz transformation  
   \begin{equation}\label{eq:vector_lorentz_transform}
       x^\mu \rightarrow x'^{\mu} = \Lambda^\mu_\nu x^\nu.
   \end{equation}
   Furthermore, a general function $f(x)$ transforms under a Lorentz transformation as $f(x)\rightarrow (f(x))' \equiv f'(x')$, where both the function its argument change.  For a vector function $V^\mu$, it undergoes a Lorentz transformation as 
   \begin{equation}\label{eq:vector-function_lorentz_transform}
       V^\mu(x) \rightarrow V'^\mu(x') = \Lambda^\mu_\nu V^\nu (x).
   \end{equation}
   With these transformation properties established, we can move on to define how Dirac spinors transform under Lorentz transformations.  We can begin by constructing a suitable $4\times 4$ matrix $S(\Lambda)$ which transforms a Dirac spinor $\psi(x)$ as in equation \ref{eq:SLambda_def}. If we then impose that $S(\Lambda)$ satisfies the equation $\gamma^\mu S(\Lambda) = S(\Lambda)\Lambda^\mu_\rho\gamma^\rho$, we can then define how the matrix $S(\Lambda)$ acts on $\bar{\psi}$ as well.
   \begin{align}\label{eq:SLambda_def}
       \psi(x) &\rightarrow \psi'(x') = \psi'(\Lambda x) = S(\Lambda)\psi(x).\\
       \bar{\psi}(x) &\rightarrow \bar{\psi}'(x') = \bar{\psi}(x)\gamma^0S^\dagger(\Lambda)\gamma^0
   \end{align}
   Incorporating all these Lorentz transformation properties together, we can see how the Dirac equation similarly transforms.  
   \begin{equation}
       \begin{split}
           \left(i\partial_\mu \gamma^\mu - m \right) \psi(x) &\rightarrow \left(i\partial'_\mu \gamma^\mu - m \right) \psi'(x)\\
           &= \left(i(\Lambda^{-1})^\nu_\mu\partial_\nu \gamma^\mu - m \right) S(\Lambda)\psi(x)\\
           &= S(\Lambda)\left(i\partial_\nu \gamma^\nu - m \right)\psi(x)
       \end{split}
   \end{equation}
   This conclusions then clearly shows that $\psi(x)$ transforms linearly under Lorentz transformations.  Therefor so long as $\psi(x)$ is a solution to the Dirac equation, $\psi'(x')$ is also a solution to the Dirac equation. Importantly, $S^\dagger(\Lambda) = \gamma^0 S^{-1}(\Lambda)\gamma^0$. \textbf{WIP FINISH POINT ABOUT $S^{-1} \neq S^\dagger$}.  We can now construct bilinear products $\bar{\psi}\Gamma\psi$ where $\Gamma$ is a $4\times4$ matrix.  $\bar{\psi}\Gamma\psi$ can then be decomposed into a set of bilinear, each has definite transformation properties under Lorentz transformation. First we can define one scalar and four vector bilinears as 
   \begin{align*}
       \bar{\psi}\psi &\rightarrow \bar{\psi}S^{-1}(\Lambda)S(\Lambda)\psi =  \bar{\psi}\psi \quad \text{(Scalar)},\\
       \bar{\psi}\gamma^\mu\psi &\rightarrow \bar{\psi}S^{-1}(\Lambda)\gamma^\mu S(\Lambda)\psi =  \Lambda^\mu_\nu(\bar{\psi}\gamma^\nu\psi) \quad \text{(Vector)}.
   \end{align*}
   Furthermore if we define the tensor $\Sigma^{\mu\nu} = \frac{1}{4}\left[\gamma^\mu,\gamma^\nu\right]$, we can then define six tensor bilinears as 
   \begin{equation*}
       \bar\psi\Sigma^{\mu\nu}\psi \rightarrow \bar{\psi}S^{-1}(\Lambda)\Sigma^{\mu\nu}S(\Lambda)\psi = \Lambda^\mu_\rho\Lambda^\nu_\sigma(\bar{\psi}\Sigma^{\rho\sigma}\psi) \quad \text{(Tensor)}.
   \end{equation*}
   Finally, we can construct 5 more bilinears, one pseudo-scalar and 4 axial-vector, if we first define an auxiliary fifth gamma matrix $\gamma^5 = i\gamma^0\gamma^1\gamma^2\gamma^3$.  This matrix has the properties $(\gamma^5)^2 = I_4$, $\{\gamma^5, \gamma^\mu\}=0$, $(\gamma^5)^\dagger = \gamma^5$.  We can therefor write
   \begin{align*}
       \bar{\psi}\gamma^5\psi &\rightarrow \bar{\psi}S^{-1}(\Lambda)\gamma^5S(\Lambda)\psi =  \mathrm{det}(\Lambda)\bar{\psi}\gamma^5\psi \quad \text{(Pseudo-scalar)},\\
        \bar{\psi}\gamma^5\gamma^\mu\psi &\rightarrow \bar{\psi}S^{-1}(\Lambda)\gamma^5\gamma^\mu S(\Lambda)\psi =  \mathrm{det}(\Lambda)\Lambda^\mu_\nu(\bar{\psi}\gamma^5\gamma^\nu\psi) \quad \text{(Axial-vector)}.
   \end{align*}
   In summary we now have a set of sixteen linearly independent matrices $I_4$ (1 scalar), $\gamma^5$ (1 pseudo-scalar), $\gamma^\mu$ (4 vectors), $\gamma^5\gamma^\mu$ (4 axial-vectors), $\Sigma^{\mu\nu}$ (6 tensors).  Any bilinear $\bar{\psi}\Gamma\psi$ can be written as the sum of terms with definite transformation properties.


    \subsection{Field interactions}\label{sec:field_ineractions}
    \textbf{Some amount of preamble...} Now let us begin to consider how particles interact. Supposed this interaction takes place at some time $t$ where $-\infty < t< \infty$.  For the multi-particle field  or state that captures this interaction $\phi(x)$, we can write that in the limit that $t\rightarrow \infty$ or $t\rightarrow -\infty$, $\phi(x) = \phi_\mathrm{out}(x)$ or $\phi(x) = \phi_\mathrm{in}(x)$ respectively. The states $\phi_\mathrm{in,out}$ are then the asymptotic limits of the Heisenberg operator $\phi$ (\textbf{wheres this come form?}), and both satisfy the Klein-Gordon equation $(\Box + m^2)\phi_\mathrm{in,out}(x) = 0$.

    For free fields, these operators can be described as plane waves:
    \begin{equation}\label{eq:plane_wave_in_out}
        \phi_\mathrm{in,out}(x) = \int \frac{d^3k}{(16\pi^3 E(k)} \left( e^{i\hat{k}\cdot\hat{x}} a^\dagger_\mathrm{in,out}(k)+ e^{-i\hat{k}\cdot\hat{x}}a_\mathrm{in,out}(k)\right),
    \end{equation}
    where $a^\dagger(k)$ and $a(k)$ are creation and annihilation operators respectively.  These operators have the properties $\ket{{k}} = a^\dagger({k)}\ket{0}$, $\bra{{k}}\ket{{k'}} = \bra{0}a(k)a^\dagger(k')\ket{0} = 16\pi^3E(k)\delta^3(k-k')$.  With these in hand we can introduce the unitary matrix operator $S$ which relates our $\phi_\mathrm{in}$ and $\phi_\mathrm{out}$ states:
    \begin{align*}
        &\phi_\mathrm{in}(x) = S \phi_\mathrm{out}(x)S^\dagger\\
        \Rightarrow& \ket{\mathrm{in}} = S\ket{\mathrm{out}}, \ket{\mathrm{out}} = S^\dagger\ket{\mathrm{in}}\\
        \Rightarrow& a^\dagger_\mathrm{in} = Sa^\dagger_\mathrm{out}S^{-1}.
    \end{align*}
    The information of the interaction therefor is contained in the S matrix.  For an initial state $i$ and final state $f$, we can write the transition amplitude $S_{fi}$ as 
    \begin{equation}\label{eq:Sfi_trans_amp}
        \braket{f,\mathrm{out}|i,\mathrm{in}} = \bra{f,\mathrm{out}}S\ket{i,\mathrm{out}} = \bra{f,\mathrm{in}}S\ket{i,\mathrm{in}} \equiv S_{fi}.
    \end{equation}

    


\section{Standard Model fields}\label{sec:standard_model_fields}
    \subsection{Gauge field theories}\label{sec:gause_field_theories}
    Discussion of Abelian vs non Abelian gauge theories
    \subsubsection{Quantum Electrodynamics}\label{sec:QED}
    \subsubsection{Quantum Chromodynamics}\label{sec:QCD}
    \subsection{The Electroweak sector}\label{sec:electroweak}
    \subsection{The Higgs sector}\label{sec:Higgs}
    \subsection{A catalog of the Standard Model}\label{sec:catalog_of_standard_model}
    This section ought to combine the various bases discussed in the previous chapter to present the combined field representation of $SU(3)_c\times SU(2)_L\times U(1)_Y$.
    

\section{Weak decays}\label{sec:weak_decays}
Though the electroweak sector would be discussed in the previous sections, the various forms of weak decays are what constitute the foundation of my research, and so an expanded discussion of their nature is warranted.  This would include the CKM matrix, flavor changing charged/neutral currents.  More specific emphasis ought to be made on standard semileptonic decays.
\subsection{Flavor changing charged currents}
\subsection{The CKM Matrix}\label{sec:CKM_matrix}
    The CKM matrix $V_\mathrm{CKM}$, with elements $V_{ij}$, put simply, parametrizes the relative strength of flavor changing weak interactions of free quarks.  In other words, it represents the probability of an up or down-type quark to decay into an opposite-type quark of a certain generation.  
    \begin{equation}\label{eq:CKM_matrix}
        V_{\mathrm{CKM}} \equiv 
        \begin{pmatrix}
         V_{ud} & V_{us} & V_{ub} \\   
         V_{cd} & V_{cs} & V_{cb} \\  
         V_{td} & V_{ts} & V_{tb} 
        \end{pmatrix}
        =
        \begin{pmatrix}
            0.97367(32) & 0.22431(85) & 0.00382(20) \\
            0.221(4)    & 0.975(6)    & 0.0411(12) \\
            0.0086(2)   & 0.0415(9)   & 1.010(27)
        \end{pmatrix}
    \end{equation}
    Considering the unitary nature of this matrix, we can construct a \textit{unitary triangle} whose angles $\phi_{1,2,3}$ are determined from the the elements of $V_{\mathrm{CKM}}$.  With this unitary constraint where $V_{ud}V^*_{ub} + V_{cd}V^*_{cb} + V_{td}V^*_{tb} = 0$, we can write
    \begin{equation}\label{eq:CKM_angles}
        \phi_1 = \arg\left(\frac{V_{cd}V^*_{cb}}{ V_{td}V^*_{tb}}\right),
        \quad \phi_2 = \arg\left(\frac{V_{td}V^*_{tb}}{ V_{ud}V^*_{ub}}\right),
         \quad \phi_3 = \arg\left(\frac{V_{ud}V^*_{ub}}{ V_{cd}V^*_{cb}}\right)
    \end{equation}
    \begin{figure}
        \centering
        \includegraphics[width=0.7\linewidth]{chapter2/pdfs/belle_rhoeta_large.pdf}
        \caption{The CKM matrix represented by as the Unitary Triangle, whose angles are defined in equation \ref{eq:CKM_angles}.  Image created by the CKMfitter group \cite{Charles_2005}.}
        \label{fig:CKM_unitary_triangle}
    \end{figure}
\subsection{Flavor changing neutral currents}
\section{The path integral and correlation functions}\label{sec:path_integrals_and_corrfuncs}
Descretizing the path integral formalism is the basis on which much of our lattice QCD work is built, and so it warrants discussion.  I would introduce the Lehman-Symanzik-Zimmerman (LSZ) reduction formula, followed by an introduction to two and thee-point correlation functions.
